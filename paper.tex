\documentclass[12pt,psamsfonts]{article}
\usepackage{amsfonts}
\usepackage{amsmath}
\usepackage{amssymb}
\usepackage{amsthm}
\usepackage{graphicx}
\usepackage{blkarray}
\usepackage{upquote}
\usepackage{tikz}
\usepackage{bbm}
\usepackage{url}
\usepackage{tikz-cd}
\usepackage{enumerate}
\usepackage{rotating}

\newcommand{\leftshift}{\,\texttt{<<}\,}
\newcommand{\downshift}{\mathbin{\rotatebox[origin=c]{90}{\leftshift}}}
\newcommand{\rightshift}{\mathbin{\rotatebox[origin=c]{180}{\leftshift}}}
\newcommand{\upshift}{\mathbin{\rotatebox[origin=c]{270}{\leftshift}}}

\newcommand{\deriv}[2]{\frac{\mathrm{d}#1}{\mathrm{d}#2}}
\newcommand{\pderiv}[2]{\frac{\partial #1}{\partial #2}}
\DeclareMathOperator{\spn}{span}
\DeclareMathOperator{\stab}{stab}
\DeclareMathOperator{\ord}{ord}
\DeclareMathOperator{\Ind}{Ind}
\DeclareMathOperator{\Res}{Res}
\DeclareMathOperator{\Image}{Image}
\DeclareMathOperator{\cof}{cof}
\DeclareMathOperator{\lcm}{lcm}
\DeclareMathOperator{\real}{Re}
\DeclareMathOperator{\arccot}{arccot}
\DeclareMathOperator{\supp}{supp}
\DeclareMathOperator{\atan2}{atan2}
\DeclareMathOperator{\interior}{int}
\DeclareMathOperator{\Br}{Br}
\DeclareMathOperator{\End}{End}
\DeclareMathOperator{\Aut}{Aut}
\DeclareMathOperator{\Gal}{Gal}
\DeclareMathOperator{\inv}{inv}
\DeclareMathOperator{\GL}{GL}
\DeclareMathOperator{\SL}{SL}
\DeclareMathOperator{\gl}{\mathfrak{gl}}
\DeclareMathOperator{\spl}{\mathfrak{sl}}
\DeclareMathOperator{\ad}{ad}
\DeclareMathOperator{\irrrep}{IrrRep}
\DeclareMathOperator{\Irr}{Irr}
\DeclareMathOperator{\Hom}{Hom}
\DeclareMathOperator{\St}{St}
\DeclareMathOperator*{\argmax}{arg\,max}
\DeclareMathOperator*{\argmin}{arg\,min}
\newcommand{\opp}{\mathrm{opp}}
\newcommand{\un}{\mathrm{un}}
\newcommand{\al}{\mathrm{al}}
\newcommand{\dd}{\mathrm{d}}
\newcommand{\comment}{}

\usepackage{hyperref}
\usepackage[capitalize,nameinlink,noabbrev]{cleveref} % to emulate \autoref style
% \usepackage{fullpage}

\newtheorem{theorem}{Theorem}[section]
\newtheorem{lemma}[theorem]{Lemma}
\newtheorem{definition}[theorem]{Definition}
\newtheorem{corollary}[theorem]{Corollary}
\newtheorem{remark}[theorem]{Remark}

\bibliographystyle{plain}
\usepackage{nicematrix}

\author{
UROP+ Final Paper, Summer 2024\\
Owen Conoly\\
Mentor: Haoshuo Fu\\
Project Suggested by: Haoshuo Fu
}
\date{}
\title{Irreducible Components and Dimension of the Springer Fiber of a Hook-Type Slodowy Slice}

% email to andre when done
\begin{document}

\maketitle
\begin{abstract}
    Let \(n \geq 1\), and let \(e\) be a regular nilpotent element of \(\gl_n\).
    Let \(m \geq 0\), and consider the Slodowy slice \(S\) at the element \((0, e) \in \gl_m \times \gl_n \subseteq \gl_{m + n}\).
    We define the \emph{Springer fiber of S} at a nilpotent \(X \in \gl_m\) as consisting of the following data: an element \(Y \in S\) which projects to \(X\) coordinate-wise, along with an element of the usual Springer fiber at \((X, Y) \in \gl_m \times \gl_{m + n}\).
    The main result of this paper is finding the (equidimensional) irreducible components of a Springer fiber of \(S\).
    To do this, we use a well-known result classifying the irreducible components of the usual Springer fiber at a nilpotent element of \(\gl_m\).
    We then use our result to find the (equidimensional) irreducible components of the variety consisting of pairs \((X, \mathcal{F})\), where \(X \in \gl_m\) is nilpotent and \(\mathcal{F}\) is an element of the Springer fiber of \(S\) at \(X\).
    We conjecture that there is a correspondence, akin to the geometric RSK correspondence, between irreducible components of this last variety and certain pairs of standard Young tableaux. 
\end{abstract}

\tableofcontents

\section{Introduction}
Let \(m \geq 0, n \geq 1\), and define \(\mathfrak{g} = \gl_m \times \gl_{m + n}\).
Let \(\mathcal{N}\) be the nilpotent cone in \(\mathfrak{g}\), and let \(\widetilde{\mathcal{N}} \to \mathcal{N}\) be the Springer resolution.
\par Let \((e, h, f)\) be a principal \(\spl_2\)-triple in \(\gl_n\).
Let \((E, H, F)\) be the \(\spl_2\)-triple in \(\gl_{m + n}\) which is the image of \((e, h, f)\) via the embedding \(\gl_n \hookrightarrow \gl_m \times \gl_n \hookrightarrow \gl_{m + n}\).
Let \(S\) be the Slodowy slice \(E + \mathfrak{z}_{\gl_{m + n}}(F)\).
We have a map \(\gl_{m + n} \to \gl_m\) given by coordinate projection.
Restricting to \(S\), we obtain a projection \(p' : S \to \gl_m\).
(It turns out that \(p'\) is surjective.)
Then, define \(p : S \to \mathfrak{g}\) by \(x \mapsto (p'(x), x)\).
Taking the map \(p : S \to \mathfrak{g}\) and the Springer resolution \(\widetilde{\mathcal{N}} \to \mathfrak{g}\), we obtain a fibred product \(S \times_\mathfrak{g} \widetilde{\mathcal{N}}\).
The variety \(S \times_\mathfrak{g} \widetilde{\mathcal{N}}\) is also discussed in \cite{sxn}[3].
\par In this paper we study \(S \times_\mathfrak{g} \widetilde{\mathcal{N}}\).
As a step towards studying \(S \times_\mathfrak{g} \widetilde{\mathcal{N}}\), we let \(\mathcal{N}_m\) be the nilpotent cone in \(\gl_m\), and we consider the map \(\pi : S \times_\mathfrak{g} \widetilde{\mathcal{N}} \to \mathcal{N}_m\) given by taking the map to \(\mathfrak{g}\) and then projecting to \(\gl_m\).
Given \(X \in \mathcal{N}_m\), we call \(\pi^{-1}(X)\) the \emph{\(n\)-Slodowy-slice Springer fiber at \(X\)}.
\par The main result of this paper is a classification of the irreducible components of an \(n\)-Slodowy-slice Springer fiber.
Then we use this result to determine the irreducible components of \(S \times_\mathfrak{g} \widetilde{\mathcal{N}}\).

\subsection{Future Work}
We describe a possible extension of the result of this paper, as well as an application to representation theory.
\par The \emph{Steinberg variety} of a Lie algebra \(\mathfrak{h}\) is \(\St_\mathfrak{h} = \widetilde{\mathcal{N}}_\mathfrak{h} \times_\mathfrak{h} \widetilde{\mathcal{N}}_\mathfrak{h}\), where the map \(\widetilde{\mathcal{N}}_\mathfrak{h} \to \mathfrak{h}\) is the Springer resolution.
There is a bijective correspondence between the Weyl group of \(\gl_m\) and the set \(\Irr(\St_{\gl_m})\) of irreducible components of \(\St_\mathfrak{\gl_m}\)(see \cite{ehf}[3.6] for details).
The Weyl group of \(\gl_m\) is the symmetric group \(S_m\).
\par There is a bijection between the set of irreducible components of \(\St_{\gl_m}\)---which we denote by \(\Irr(\St_{\gl_m})\)---and the set
\[T_m = \{(t_1, t_2) : \lambda \vdash m; t_1, t_2 \in \mathrm{SYT}(\lambda)\}.\]
(We write \(\mathrm{SYT}(\lambda)\) to denote the set of standard Young tableaux of shape \(\lambda\).)
One way to obtain the bijection is as follows.
First take the aforementioned correspondence between \(\St_{\gl_m}\) and the Weyl group \(S_m\).
Then take the Robinson-Schensted-Knuth (RSK) correspondence, which gives a bijection between \(S_m\) and \(T_m\).
Putting these together, we get a correspondence \(\St_{\gl_m} \leftrightarrow T_m\).
A more direct and illuminating explanation of the correspondence \(\St_{\gl_m} \leftrightarrow T_m\) is given by the geometric RSK correspondence, as described in \cite{geomRSK}.
\par An element of \(\St_{\gl_m}\) is a triple \((X, \mathfrak{b}_1, \mathfrak{b}_2)\), where \(X \in \mathcal{N}_m\), and \(\mathfrak{b}_1, \mathfrak{b}_2\) are elements of the Springer fiber at \(X\).
When the Jordan form of \(X\) has shape \(\lambda\), the geometric RSK correspondence sends this triple to a pair \((t_1, t_2)\), where both are of shape \(\lambda\).
\par Somewhat similarly, we will show that \(S \times_\mathfrak{g} \widetilde{\mathcal{N}}\) is a subset of the set of tuples \((X, Y, \mathfrak{b}_1, \mathfrak{b}_2)\), where \(X \in \mathcal{N}_m, Y \in \mathcal{N}_{m + n}\), and \(\mathfrak{b}_1\) (resp. \(\mathfrak{b}_2\)) is an element of the Springer fiber at \(X\) (resp. \(Y\)).
\par For any given relation \(R \subseteq \{\lambda : \lambda \vdash m\} \times \{\mu : \mu \vdash (m + n)\}\), we can define 
\[T_{m,n} = \{(t_1, t_2) : \lambda \vdash m; \mu \vdash (m + n); t_1 \in \mathrm{SYT}(\lambda); t_2 \in \mathrm{SYT}(\mu); R(\lambda, \mu)\}.\]
We conjecture that, for some appropriate choice of the relation \(R\), there is a correspondence, analogous to \(\St_{\gl_m} \leftrightarrow T_m\), between \(S \times_\mathfrak{g} \widetilde{\mathcal{N}}\) and \(T_{m,n}\).
A natural extension of this project would explore this correspondence.
\par Given an element of \(\Irr(\St_{\gl_m})\), we can first map it to the corresponding \((t_1, t_2) \in T_m\), and then obtain an element of \(\End(\mathbb{C}[\mathrm{SYT}(\lambda)])\) by taking the map which sends \(t_1 \mapsto t_2\) and \(t' \mapsto 0\) for \(t' \in \mathrm{SYT}(\lambda) \setminus \{t_1\}\).
In this way, we obtain a \(\mathbb{C}\)-algebra isomorphism
\[\mathbb{C}[\Irr(\St_{\mathfrak{\gl_m}})] \cong \bigoplus_{\lambda \vdash m} \End(\mathbb{C}[\mathrm{SYT}(\lambda))).\]
This is an instance of the fact that for any finite group \(G\), we have \(\mathbb{C}[G] \cong \bigoplus_{V \in \irrrep(G)} \End(V)\).
\par If we had a correspondence \(S \times_\mathfrak{g} \widetilde{\mathcal{N}} \leftrightarrow T_{m,n}\), this could give an analogous isomorphism 
\[\mathbb{C}[\Irr(S \times_\mathfrak{g} \widetilde{\mathcal{N}})] \cong \bigoplus_{(\lambda, \mu) \in R} \Hom(\mathbb{C}[\mathrm{SYT}(\lambda)], \mathbb{C}[\mathrm{SYT}(\mu)]),\]
which would be an isomorphism of \(\mathbb{C}[S_m]\)- or \(\mathbb{C}[S_{m + n}]\)-modules rather than \(\mathbb{C}\)-algebras.
(To give the RHS the structure of a \(\mathbb{C}[S_m]\)-module, we just view \(\mathbb{C}[\mathrm{SYT}(\lambda)]\) as a representation of \(S_m\) in the standard way.)

\subsection{Overview}
\par \cref{prelim} reviews some preliminary material.
\cref{principal_sl2} finds the unique (up to similarity) principal \(\spl_2\)-triple in \(\gl_n\).
\cref{find_sxn} embeds this \(\spl_2\)-triple into \(\gl_{m + n}\) as described above.
We compute the Slodowy slice and end up with a nice description of \(S \times_\mathfrak{g} \widetilde{\mathcal{N}}\).
\par \cref{p_comp_strat} discusses how to reduce the problem of finding the irreducible components of an \(n\)-Slodowy-slice Springer fiber at \(X \in \mathcal{N}_m\) to an easier problem.
\cref{q_comp} solves the easier problem.
\cref{p_comp} finds the irreducible components of an \(n\)-Slodowy-slice Springer fiber.
\cref{sxn_comps} applies the results of \cref{p_comp} to find the irreducible components of \(S \times_\mathfrak{g} \widetilde{\mathcal{N}}\).
Finally, \cref{linalg} proves some linear algebra lemmas that were used in the paper.

\subsection{Acknowledgments}
Thank you to Haoshuo Fu for suggesting the fun project, guiding me through it, and helping me learn many things, both by talking to me about them and by suggesting things to read.
Thank you also to the organizers of the MIT math department's UROP+ program, and to MIT for providing funding.

\section{Preliminary Definitions and Facts}\label{prelim}
\subsection{Conventions and Notations}
We write \(\GL_m, \spl_m\) to denote \(\GL_m(\mathbb{C}), \spl_m(\mathbb{C})\), and so on.
The notation \(\mathfrak{z}_\mathfrak{h}(X)\) denotes the centralizer of \(X\) in the Lie algebra \(\mathfrak{h}\).
\par By \(J_m\) we refer to the nilpotent \(m \times m\) Jordan block (which has ones \emph{below} the diagonal).
Given a partition \(\lambda = (\lambda_1, ..., \lambda_\ell)\), we write \(J(\lambda)\) to denote the block matrix
\[\begin{pmatrix}
    J_{\lambda_1} \\
    & \ddots \\
    & & J_{\lambda_\ell}
\end{pmatrix}.\]
A partition is always indexed in nonincreasing order, even if it is written down differently when defined.
For example, if \(\mu = (4, 6, 3)\), then \((\mu_1, \mu_2, \mu_3) = (6, 4, 3)\).
\par Throughout the paper, when a nilpotent operator \(X : \mathbb{C}^m \to \mathbb{C}^m\) of shape \(\lambda = (\lambda_1, ..., \lambda_\ell) \vdash m\) is in the context, we write \((e_{ij})_{1 \leq i \leq \ell, 1 \leq j \leq \lambda_i}\) to denote a Jordan basis of \(\mathbb{C}^m\) with the property that for all \(i,j\) we have \(Xe_{ij} = e_{i,j - 1}\).
(By convention, \(e_{ij} := 0\) for \(j < 1\), so that condition makes sense.)
Also, we write \(f_n, ..., f_1\) to denote the basis of \(\mathbb{C}^n\) with the property that \(J_n f_i = f_{i - 1}\).

\subsection{Springer Fibers}
Let \(\mathfrak{h}\) be a Lie algebra, let \(\mathcal{N}_\mathfrak{h} \subseteq \mathfrak{h}\) be the nilpotent cone in \(\mathfrak{h}\), and let \(\mathcal{B}_\mathfrak{h}\) be the variety of Borel subalgebras of \(\mathfrak{h}\).
Let \(\widetilde{\mathcal{N}}_\mathfrak{h} = \{(\mathfrak{b}, n) \in \mathcal{B}_\mathfrak{h} \times \mathcal{N}_\mathfrak{h} : n \in \mathfrak{b}\}\).
Let \(\pi_\mathfrak{h} : \widetilde{\mathcal{N}}_\mathfrak{h} \to \mathcal{N}_\mathfrak{h}\) be the projection onto the second coordinate.
We call \(\pi_\mathfrak{h}\) the \emph{Springer resolution}.
For \(n \in \mathcal{N}_\mathfrak{h}\), we call \(\pi_\mathfrak{h}^{-1}(n)\) the \emph{Springer fiber at \(n\)}.

\subsection{Springer Fibers in \texorpdfstring{\(\gl_m\)}{gl\_m}}
Let \(\mathcal{N}_m\) be the nilpotent cone in \(\gl_m\), and let \(\mathcal{B}_m\) be the variety of Borel subalgebras of \(\gl_m\).
Let \(\mathfrak{h} \subseteq \gl_m\) be the subalgebra of upper triangular matrices.
The variety of Borel subalgebras of \(\gl_m\) is \(\mathcal{B}_m = \{g\mathfrak{h}g^{-1} : g \in \GL_m\}\).
Thus, the Springer fiber at \(X \in \mathcal{N}_m\) is 
\[\mathcal{F}_X = \{g\mathfrak{h}g^{-1} : g \in \GL_m; X \in g\mathfrak{h}g^{-1}\}.\]
\begin{definition}
    A \emph{flag} \(V_\bullet\) of \(\mathbb{C}^m\) is a sequence of subspaces
    \[0 = V_0 \subseteq V_1 \subseteq \cdots \subseteq V_m = \mathbb{C}^m,\]
    where \(\dim V_i = i\).
\end{definition}
We say that \(X \in \gl_m\) \emph{preserves} a flag \(V_\bullet\) if for each \(i\) we have \(XV_i \subseteq V_i\).
For \(X \in \gl_m\), we have [\(X \in \mathcal{N}_m\), and \(X\) preserves \(V_\bullet\)] if and only if [for all \(i\), \(XV_{i + 1} \subseteq V_i\)].
\par One flag is the \emph{standard flag} \(E_\bullet\), where \(E_i\) is generated by the first \(i\) standard basis vectors.
The subalgebra \(\mathfrak{h}\) is exactly the subset of \(\gl_m\) which preserves \(E_\bullet\).
So, we can think of \(\mathcal{B}_m\) as the variety of flags of \(\mathbb{C}^m\), via the correspondence 
\[g\mathfrak{h}g^{-1} \leftrightarrow g \cdot E_\bullet.\]
(\(\GL_m\) acts on the set of flags via \((g \cdot V_\bullet)_j := gV_j\).)
Note that \(X\) preserves \(g \cdot E_\bullet\) if and only if \(X \in g \mathfrak{h} g^{-1}\).
Thus, we may write the Springer fiber at \(X\) in terms of flags, as 
\[\mathcal{F}_X = \{g \cdot E_\bullet : g \in \GL_m; X \in gHg^{-1}\} = \{V_\bullet : \forall i. \; XV_{i + 1} \subseteq V_i\}.\]

\begin{theorem}[{\cite[2.1]{springer_fiber}}]\label{usual_springer_fiber}
    The irreducible components of the Springer fiber at \(J(\mu)\) are in bijection with the standard Young tableaus of shape \(\mu\).
    The irreducible components are equidimensional, of dimension \(\sum_{i < j} \min(\mu_i, \mu_j) = \sum_i (i - 1) \mu_i\).
\end{theorem}

\subsection{Slodowy Slices}
A basis for \(\spl_2\) is
\[e' := \begin{pmatrix}0 & 1 \\ 0 & 0 \\\end{pmatrix}, h' := \begin{pmatrix}1 & 0 \\0 & -1\end{pmatrix}, f' := \begin{pmatrix}0 & 0 \\1 & 0\end{pmatrix}.\]
Given a Lie algebra \(\mathfrak{h}\) and a homomorphism \(\phi : \spl_2 \to \mathfrak{h}\) sending \((e', h', f')\) to \((e, h, f)\), we say that \((e, h, f)\) is an \emph{\(\spl_2\)-triple}.
If \(\mathfrak{h}\) is semisimple, then given any nilpotent \(e \in \mathfrak{h}\), the Jacobson-Morozov theorem \cite[3.7.1]{ehf} says that there exist \(h, f \in \mathfrak{h}\) such that \((e, h, f)\) is an \(\spl_2\)-triple.
\par Given \((e, h, f)\), we define the \emph{Slodowy slice at \(e\)} as \(S_e = e + \mathfrak{z}_\mathfrak{h}(f)\).
By the Jacobson-Morozov theorem, if \(\mathfrak{h}\) is semisimple, then we can always find a Slodowy slice at any nilpotent \(e \in \mathfrak{h}\).
In particular, we can find a Slodowy slice at any nilpotent \(e \in \gl_n\), since \(\spl_n \subseteq \gl_n\) is semisimple and contains every nilpotent element of \(\gl_n\).

\section{Principal \texorpdfstring{\(\spl_2\)}{sl\_2}-triples in \texorpdfstring{\(\gl_n\)}{gl\_n}}\label{principal_sl2}
The unique nilpotent regular element of \(\gl_n\) (up to similarity) is \(J_n\).
In this section we show that, in fact, there is a unique (up to similarity) principal \(\spl_2\)-triple in \(\gl_n\).
Let \(e = J_n\).
\begin{lemma}\label{simple_sl2_triple}
    There is exactly one way to choose \(h, f \in \gl_n\) so that \((e, h, f)\) is an \(\spl_2\)-triple.
\end{lemma}
\begin{proof}
    
Note that \([h', e'] = 2e'\), and \([e', f'] = h'\), and \([h', f'] = -2f'\).
Thus \(e, h, f\) must obey the same relations.
In particular, \(he - eh = 2e\).  
The matrix \(eh\) is \(h\) shifted down one, and \(he\) is \(h\) shifted left one.
So, \([h,e] = 2e\) implies that \(h_{ij} = 0\), except when \(i = j\) or \((i,j) = (n,1)\).
It also implies that \(h_{ii} = h_{i - 1, i - 1} + 2\), and hence \(h_{ii} = h_{11} + 2(i - 1)\).
\par Similarly, \(ef - fe = h\); this implies that \(f_{ij} = 0\), except when \(j = i + 1\) or \((i,j) = (n,1)\).
Also, \(f_{i,i + 1} = f_{i + 1, i + 2} + h_{i + 1, i + 1}\), and \(f_{1,2} = -h_{1, 1}\), and \(f_{n - 1, n} = h_{n,n}\).
Putting these equations together,
\begin{align*}
    h_{nn} + h_{11} & = \\
    f_{n - 1,n} - f_{1,2} & = \\
    \sum_{i = 1}^{n - 2} (f_{i + 1, i + 2} - f_{i,i + 1}) & = \\
    - \sum_{i = 1}^{n - 2} h_{i + 1, i + 1} & .
\end{align*}
That is, \(h \in \spl_n\).
This shows that \(h_{11} = 1 - n, h_{22} = 3 - n, ..., h_{nn} = n - 1\).
\par Now that we have mostly determined \(h\) and \(f\), looking at \([e,f] = h\) and \([h,f] = -2f\) shows that \(h_{n,1} = f_{n,1} = 0\).
So we have determined \(h\); it is 
\[h = \begin{pmatrix}
    1 - n  & \\
    & 3 - n \\
    & & \ddots \\
    & & & n - 3 \\
    & & & & n - 1 
\end{pmatrix}.\]
Now, we can use our expression for \(f\) in terms of \(h\) to obtain
\[f = \begin{pmatrix}
    0 & n - 1 \\
    & 0 & (n - 1) + (n - 3) \\
    & & \ddots & \ddots \\
    & & & 0 & (n - 1) + \cdots + (3 - n) \\
    & & & & 0
\end{pmatrix} = \]
\[\begin{pmatrix}
    0 & (1) (n - 1) \\
      & 0 & (2) (n - 2) \\
      & & \ddots & \ddots & \\
    & & & 0 & (-2)(n - 2) \\
    & & & & 0 & (-1)(n - 1)\\
    & &  & & & 0
\end{pmatrix}.\]
\end{proof}

\section{Finding \texorpdfstring{\(S \times_\mathfrak{g} \widetilde{\mathcal{N}}\)}{S x\_g N}}\label{find_sxn}
\subsection{Finding the Slodowy Slice \texorpdfstring{\(S\)}{S}}
In the previous section we computed the principal \(\spl_2\)-triple \((e, h, f)\) in \(\gl_n\).
Embedding this into \(\gl_{m + n}\) as described previously, we obtain
\[(E, H, F) = \left(\begin{pmatrix}0 & 0 \\0 & e\end{pmatrix}, \begin{pmatrix}0 & 0 \\0 & h\end{pmatrix}, \begin{pmatrix}0 & 0 \\0 & f\end{pmatrix}\right).\]

\begin{lemma}\label{small_centralizer}
    \(\mathfrak{z}_{\gl_n}(f)\) is the set of upper-triangular \(X \in \gl_n\) such that for all \(i,j \in \{1,...,n-1\}\),
    \[X_{i + 1,j + 1} = \frac{j(n - j)}{i(n - i)} X_{ij}.\]
\end{lemma}
\begin{proof}
    Let \(X \in \mathfrak{z}_{\gl_n}(f)\).
    Looking at the definition of \(f\) from the previous section (and zero-padding the matrices), we see that \((fX)_{ij} = i (n - i) X_{i + 1,j}\), and \((Xf)_{ij} = (j - 1)(n - (j - 1)) X_{i, j - 1}\).
    So, for any \(i, j \in \{1,...,n\}\),
    \[i(n - i)X_{i + 1,j} = (j - 1)(n - (j - 1))X_{i, j - 1}.\]
    Taking \(j = 1\), the condition above tells us that for all \(i \geq 2\) we have \(X_{i, 1} = 0\).
    Taking \(j > 1\) and \(i < n\), we get that for all \(i,j \in \{1,...,n-1\}\),
    \[X_{i + 1, j + 1} = \frac{j(n - j)}{i(n - i)} X_{ij}.\]
    So, every \(X \in \mathfrak{z}_{\gl_n}(f)\) is upper triangular and satisfies the condition above.
    Conversely, it is clear that for such \(X\) we have \(fX = Xf\).
\end{proof}

\begin{lemma}\label{big_centralizer}
    \[\mathfrak{z}_{\gl_{m + n}}(F) =  \left\{\begin{pNiceArray}{ccc|ccc}
        & & & & & \vert \\
        & X & & & & b    \\
        & & & & & \vert \\
       \hline
       \text{---} & a & \text{---} &  \\
        &  & &  & Y  \\
        & & & & & 
       \end{pNiceArray} : X \in \gl_m; Y \in \mathfrak{z}_{\gl_n}(f); a,b \in \mathbb{C}^m \right\}.\]
\end{lemma}
\begin{proof}
    Let \(Z = \begin{pmatrix}Z_{11} & Z_{12}\\Z_{21} & Z_{22}\end{pmatrix} \in \mathfrak{z}_{m + n}(F)\).
    We have 
    \[\begin{pmatrix}0 & Z_{12}f\\0 & Z_{22} f\end{pmatrix} =  ZF = FZ = \begin{pmatrix}0 & 0\\fZ_{21} & fZ_{22}\end{pmatrix}.\]
    There is no restriction on \(Z_{11}\).
    The condition \(Z_{12}f = 0\) means that all but the last column of \(Z_{12}\) must be zero, and the condition \(0 = fZ_{21}\) means that all but the first row of \(Z_{21}\) must be zero.
    Finally, the condition \(Z_{22}f = fZ_{22}\) means that \(Z_{22} \in \mathfrak{z}_{\gl_n}(f)\).
\end{proof}

Taken together, the previous lemmas provide a nice characterization of the Slodowy slice \(S = E + \mathfrak{z}_{\gl_{m + n}}(F)\).
\begin{corollary}\label{slodowy_slice}
    \[S = \left\{\begin{pNiceArray}{ccc|ccc}
        & & & & & \vert \\
        & X & & & & b    \\
        & & & & & \vert \\
       \hline
       \text{---} & a & \text{---} &  \\
        & & &  & e + Y\\
        &  & &  &   
       \end{pNiceArray} : a, b \in \mathbb{C}^m, X \in \gl_m, Y \in \mathfrak{z}_{\gl_n}(f)\right\}.\]
\end{corollary}

\subsection{A description of \texorpdfstring{\(S \times_\mathfrak{g} \widetilde{\mathcal{N}}\)}{S x\_g N}}
Recall the map \(S \to \mathfrak{g}\) given by \(Z \mapsto (p'(Z), Z)\), where \(p' : \gl_{m + n} \to \gl_m\) is coordinate projection taking a matrix to its upper-left corner.
We also have the Springer resolution \(\widetilde{\mathcal{N}} \to \mathfrak{g}\).
From these two maps we define the fibred product \(S \times_\mathfrak{g} \widetilde{\mathcal{N}}\).
\par To obtain an explicit description of \(S \times_\mathfrak{g} \widetilde{\mathcal{N}}\), we begin by finding the image of the projection \(S \times_\mathfrak{g} \widetilde{\mathcal{N}} \to S\).
Since the image of the Springer resolution is \(\mathcal{N} = \mathcal{N}_m \times \mathcal{N}_{m + n}\), the image of the projection \(S \times_\mathfrak{g} \widetilde{\mathcal{N}} \to S\) is simply \(S' := \{Z \in S : p'(Z) \in \mathcal{N}_m, Z \in \mathcal{N}_{m + n}\}\).
\begin{lemma}
    \[S' = \left\{A_{X, a, b} := \begin{pNiceArray}{ccc|ccc}
    & & & & & \vert \\
    & X & & & & b    \\
    & & & & & \vert \\
   \hline
   \text{---} & a & \text{---} \\
    & & & & e & \\
    & & &  & 
   \end{pNiceArray} \in \mathcal{N}_{m + n}: a, b \in \mathbb{C}^m, X \in \mathcal{N}_m\right\}.\]
\end{lemma}
\begin{proof}
    Since every element of \(\mathfrak{z}_{\gl_n}(f)\) is upper triangular by \cref{small_centralizer}, \cref{bottom_right_nilp} says that if a matrix \(Z \in S\) of the form given in \cref{slodowy_slice} is nilpotent, and the upper-left block \(p'(Z) = X\) is nilpotent as well, then \(e + Y\) must be nilpotent.
    Finally, \cref{upper_triangle_zero} says that if \(e + Y\) is nilpotent for \(Y \in \mathfrak{z}_{\gl_n}(f)\), then \(Y = 0\).
    \par Thus every element of \(S'\) must simply have \(e\) in its bottom-right block.
    So, every element of \(S'\) is of the desired form.
\end{proof}
This is not a fully explicit characterization of \(S'\), since we don't say which choices of \(X\) and \(a, b \in \mathbb{C}^m\) lead to \(A_{X, a, b}\) being nilpotent.
We could use \cref{block_determinant} to find a necessary and sufficient condition on \(X, a, b\); however, the above description of \(S'\) will be good enough for our purposes.
\begin{corollary}\label{sn_iso}
    \begin{align*}
        S \times_\mathfrak{g} \widetilde{\mathcal{N}} & = \\
        \{((X, A_{X, a, b}), ((X, A_{X, a, b}), \mathfrak{b})) \in S \times \widetilde{\mathcal{N}}\} & \cong \\
        \{(A_{X, a, b}, \mathfrak{b}) : X \in \gl_m; a,b \in \mathbb{C}^m; \mathfrak{b} \in \mathcal{B}; (X, A_{X, a, b}) \in \mathfrak{b} \cap \mathcal{N}\} & .
    \end{align*}
\end{corollary}
Define \(\pi : S \times_\mathfrak{g} \widetilde{\mathcal{N}} \to \mathcal{N}_m\) by \((A_{X, a, b}, \mathfrak{b}) \mapsto X\).
We define the \emph{\(n\)-Slodowy-slice Springer fiber} at \(X \in \mathcal{N}_m\) to be the fiber \(\pi^{-1}(X)\).
Because \(\mathcal{B} = \mathcal{B}_m \times \mathcal{B}_{m + n}\),  
\[\pi^{-1}(X) \cong \{(A_{X, a, b}, \mathfrak{b}_{m + n}) : a,b \in \mathbb{C}^m; A_{X, a, b} \in \mathfrak{b}_{m + n} \cap \mathcal{N}_{m + n}\} \times \{\mathfrak{b}_m : X \in \mathfrak{b}_m\}.\]
The right factor of the product is simply the usual Springer fiber \(\mathcal{F}_X\).
\par Let 
\[\mathcal{P}_X = \{(A_{X, a, b}, V_\bullet) : \forall i. \; A_{X, a, b} V_i \subseteq V_{i - 1}\}.\]
By the correspondence between Springer fibers in \(\mathcal{B}_{m + n}\) and flags of \(\mathbb{C}^{m + n}\), the variety \(\mathcal{P}_X\) is isomorphic to the left factor of \(\pi^{-1}(X)\).
\par In the next few sections, we will find the irreducible components of \(\mathcal{P}_X\).
Then we will use this result, along with the result about the usual Springer fiber \(\mathcal{F}_X\), to find the irreducible components of \(\pi^{-1}(X)\) and then of \(S \times_\mathfrak{g} \widetilde{\mathcal{N}}\).

\section{Strategy for Finding Components of \texorpdfstring{\(\mathcal{P}_{J(\lambda)}\)}{P\_\{J(\textbackslash lambda)\}}}\label{p_comp_strat}
\par Let \(X = J(\lambda) \in \gl_m\), where \(\lambda = (\lambda_1, ..., \lambda_\ell)\).
In this section we write \(\mathcal{P} := \mathcal{P}_X\) and \(A_{a, b} := A_{X, a, b}\).
Let \((e_{ij})_{1 \leq i \leq k, 1 \leq j \leq \lambda_i}\) be the standard basis for \(\mathbb{C}^m\).
Let \((f_j)_{1 \leq j \leq k}\) be the standard basis for \(\mathbb{C}^n\).
We index the bases such that \(Xe_{ij} = e_{i,j - 1}\), and \(J_n f_j = f_{j - 1}\).
\par For \(1 \leq w \leq \ell\) and \(0 \leq r \leq \lambda_w\) (note that we allow \(r = 0\); recall \(e_{i0} = 0\)), define \(\mathcal{P}_{w,r}\) as the set of \((A_{a,b}, V_\bullet) \in \mathcal{P}\) such that there exist \(P \in \GL_m\) and \(b' \in \mathbb{C}^n\) such that \((P^{-1}, I_n) A_{a, b} (P, I_n) = A_{e_{wr}, b'}\).

\begin{lemma}\label{pwr_union}
    \(\mathcal{P} = \bigcup_{1 \leq w \leq k, 0 \leq r \leq \lambda_{w}} \mathcal{P}_{wr}\).
\end{lemma}
\begin{proof}
    For any \(a \in \mathbb{C}^m\), define the linear map \(\phi_a : \mathbb{C}^{m + n} \to \mathbb{C}\) by \(e_{ij} \mapsto a_{ij}\) and \(f_j \mapsto 0\).
    Note that \(A_{a,b}\) is the unique linear map \(\mathbb{C}^{m + n} \to \mathbb{C}^{m + n}\) that sends \(e_{ij}\) to \(e_{i,j-1} + \phi_a(e_{ij}) f_n\), sends \(f_{i + 1}\) to \(f_i\), and sends \(f_1\) to \((b, 0)\).
    \par Let \((A_{a, b}, V_\bullet) \in \mathcal{P}\).
    Since \(X\) is nilpotent, \cref{normalization} says that there is a change of basis \(P : \mathbb{C}^m \to \mathbb{C}^m\) such that the vectors \(Pe_{ij}\) form a Jordan basis for \(X\), and for all but one pair \((i, j)\) we have \(\phi_a(Pe_{ij}) = 0\).
    \par Since \((Pe_{ij})_{ij}\) is a Jordan basis, we have \((P^{-1}, I_n) A_{a,b}(P, I_n)\) for some \(a',b' \in \mathbb{C}^m\).
    All that is left is to show that \(a'\) is of the form \(e_{wr}\) for some \(w\) and \(r\).
    Indeed, this results from the fact that \(\phi_{a'}(e_{ij}) = \phi_a(Pe_{ij}) = 0\) for all but one pair \((i,j)\).
\end{proof}
\begin{lemma}
    \(\mathcal{P}_{w_1r_1} = \mathcal{P}_{w_2r_2}\) exactly when either \(r_1 = r_2 = 0\), or \((\lambda_{w_1}, r_1) = (\lambda_{w_2}, r_2)\).
\end{lemma}
\begin{proof}
    If \(r_1 = r_2 = 0\), we have \(e_{wr_1} = e_{wr_2} = 0\), so by definition, \(\mathcal{P}_{wr_1} = \mathcal{P}_{wr_2}\).
    And if \((\lambda_{w_1}, r_1) = (\lambda_{w_2}, r_2)\), then any matrix of the form \(A_{e_{w_1r_1}b'}\) can be transformed to a matrix of the form \(A_{e_{w_2r_2}b'}\) by making the change of basis that swaps \(e_{w_1j}\) with \(e_{w_2j}\).
    \par Conversely, suppose \(\mathcal{P}_{w_1r_1} = \mathcal{P}_{w_2r_2}\).
    Clearly either \(r_1 = r_2 = 0\) or \(r_1 \neq 0\) and \(r_2 \neq 0\).
    In the first case, we're done.
    \par Otherwise, let \(k \geq 0\) be minimal such that there exist \(u,v \in \mathbb{C}^m\) with \(A_{e_{w_1r_1},0} (u,0) = (v,0) + f_n\) and \(A_{e_{w_1r_2},0}^k (v,0) = 0\).
    Obviously \(k = r_1 - 1\).
    Since (by hypothesis) \(A_{e_{w_1r_1},0}\) and \(A_{e_{w_2r_2},0}\) differ only by a change of basis of \(\mathbb{C}^m\), the same reasoning shows that \(k = r_2 - 1\).
    Hence \(r_1 = r_2\).
    Now we only need to show that \(\lambda_{w_1} = \lambda_{w_2}\).
    \par Let \(k' \geq 0\) be maximal such that there exist \(u,v \in \mathbb{C}^m\) with \(A_{e_{w_1r_1},0}^{k'} (u,0) = (v,0) + f_n\).
    Obviously \(k' = \lambda_{w_1} - r_1 + 1\).
    Since \(k'\) is independent of \(\mathbb{C}^m\)-basis, we get \(\lambda_{w_1} - r_1 + 1 = k' = \lambda_{w_2} - r_2 + 1\), and therefore \(\lambda_{w_1} = \lambda_{w_2}\).
\end{proof}

\begin{lemma}
    When \(\mathcal{P}_{w_1r_1} \neq \mathcal{P}_{w_2r_2}\), we have \(\mathcal{P}_{w_1r_1} \cap \mathcal{P}_{w_2r_2} = \emptyset\). 
\end{lemma}
\begin{proof}
    It suffices to remark that \(\{(x,y) : \exists w,r. \; x,y \in \mathcal{P}_{wr}\}\) is an equivalence relation.
    Reflexivity is the statement of \cref{pwr_union}, and symmetry and transitivity are obvious from the definition of \(\mathcal{P}_{wr}\).
\end{proof}

Now, fix any \(w\) and \(r\).
We will find the irreducible components of \(\mathcal{P}_{w,r}\).
These will all happen to be equidimensional (with dimensions independent of \(w\) and \(r\)), so their closures in \(\mathcal{P}\) will be irreducible components of \(\mathcal{P}\).
\par Let 
\[G = \{P \in \GL_m : P X P^{-1} = X\},\]
and
\[G_{wr} = \{A \in G : e_{wr}A = e_{wr}\}.\]
Also, define 
\[\mathcal{Q}_{wr} = \{(A_{e_{wr}, b}, V_\bullet) \in \mathcal{P}_{wr}\}.\]
Let \(G\) act on \(\mathcal{P}_{wr}\) by
\[P \cdot (A_{a,b}, V_\bullet) := ((P, I_n) A_{a, b} (P, I_n)^{-1}, (P, I_n)V_\bullet) = (A_{a P^{-1}, Pb}, (P, I_n) V_\bullet).\]
Note that for any \(x \in \mathcal{Q}_{wr}\), we have \(G_{wr} = \{g \in G : g \cdot x \in \mathcal{Q}_{wr}\}\).
So, by restriction of \(G\) to \(G_{wr}\) and \(\mathcal{P}_{wr}\) to \(\mathcal{Q}_{wr}\), we obtain an action of \(G_{wr}\) on \(\mathcal{Q}_{wr}\).
\par Consider the map \(\varphi_{wr} : \mathcal{Q}_{wr} \times G \to \mathcal{P}_{wr}\) defined by
\[(x, P) \mapsto P \cdot x.\]
Letting \(G_{wr}\) act on \(G\) by \(g \cdot h := hg^{-1}\), we obtain an action of \(G_{wr}\) on \(\mathcal{Q}_{wr} \times G\).

\begin{lemma}\label{principal_bundle}
    The map \(\varphi_{wr}\) is a principal \(G_{wr}\)-bundle.
\end{lemma}
\begin{proof}
    We need to show that \(G_{wr}\) acts freely and transitively on the fibers of \(\varphi_{wr}\).
    It is obvious that \(G_{wr}\) acts freely on \(\mathcal{Q}_{wr} \times G\); it is enough to note that it acts freely on \(G\).
    Now we check that it acts transitively.
    \par Let \(y \in \mathcal{P}_{wr}\).
    By definition of \(\mathcal{P}_{wr}\), there is \(P_y \in G\) with \(P_y \cdot y \in \mathcal{Q}_{wr}\).
    \begin{align*}
        \varphi_{wr}^{-1}(y) = \{(x, P) : P \cdot x = y \} & = \\
        \{(P^{-1} y, P) : P^{-1} y \in \mathcal{Q}_{wr} \} & = \\
        \{((P^{-1}P_y^{-1}) \cdot (P_y \cdot y), P) : (P^{-1} P_y^{-1}) \cdot (P_y \cdot y) \in \mathcal{Q}_{wr}\} & .
    \end{align*}
    Since \(G_{wr} = \{g \in G : g \cdot (P_y \cdot y) \in \mathcal{Q}_{wr}\}\), the expression above becomes
    \[\{((P^{-1}P_y^{-1}) \cdot (P_y \cdot y), P) : (P^{-1} P_y^{-1}) \in G_{wr}\}.\]
    Setting \(Q := P^{-1} P_y^{-1}\), and observing that \(P = P_y^{-1} Q^{-1}\), the above becomes
    \begin{align*}
        \{(Q \cdot (P_y \cdot y), P_y^{-1} Q^{-1}) : Q \in G_{wr}\} & = \\
        \{Q \cdot (P_y \cdot y, P_y^{-1}) : Q \in G_{wr}\} & .
    \end{align*}
    Thus the fibers are exactly the \(G_{wr}\)-orbits; or in other words, \(G_{wr}\) acts transitively on the fibers, as desired.
\end{proof}
Next, we will find the irreducible components \(X \subseteq \mathcal{Q}_{wr}\), and we then argue that the irreducible components of \(\mathcal{P}_{wr}\) are of the form \(\varphi_{wr}(X \times G)\).
\par Actually, \(\mathcal{Q}_{wr}\) is unnecessarily difficult to think about; it is easiest in the case \(r = 0\).
So, we change basis to make \(r = 0\).
Let \(m' = m - r\), and \(n' = n + r\).
Let \(\lambda' = (\lambda_1, ..., \lambda_{w - 1}, r - 1, \lambda_{w + 1}, ..., \lambda_\ell)\).
Let \(X' = J(\lambda')\).
Let \(\mathcal{Q}' = \{(A'_{X', 0, b}, V_\bullet) : \forall i. \; A'_{X', 0, b} V_{i + 1} \subseteq V_i\}\), where
\[A'_{X',a,b} := \begin{pNiceArray}{ccc|ccc}
    & & & & & \vert \\
    & X' & & & & b    \\
    & & & & & \vert \\
   \hline
   \text{---} & a & \text{---} \\
    & & & & J_{n'} & \\
    & & &  & 
   \end{pNiceArray} \in \gl_{m' + n'}\]
\begin{lemma}\label{a_zero}
    \(\mathcal{Q}_{wr} \cong \mathcal{Q}'\).
\end{lemma}
\begin{proof}
    Let \((e_{ij}')_{1 \leq i \leq \ell, 1 \leq j \leq \lambda_i - \delta_{jw} r}\) be the standard Jordan basis for \(X'\).
    (The \(\delta\) is the Kronecker \(\delta\).)
    Let \(f_1', ..., f_{n'}'\) be a basis for \(\mathbb{C}^{n'}\) such that \(A_{X', 0, b} f_{i + 1}' = f_i'\), and \(A_{X', 0, b} f_1 = (b, 0)\).
    Define the linear map \(Q_{wr} : \mathbb{C}^{m' + n'} \to \mathbb{C}^{m + n}\) by:
    \begin{itemize}
        \item For all \(i,j\), \(e_{ij}' \mapsto e_{ij}\).
        \item For \(j = 1, ..., r\), \(f_{n + j}' \mapsto e_{w,\lambda_w + j}\).
        \item For \(j = 1, ..., n'\), \(f_j' \mapsto f_j + e_{w, \lambda_w - n' + j}\).
    \end{itemize}
    % The inverse map \(Q_{wr}^{-1} : \mathbb{C}^{m + n} \to \mathbb{C}^{m' + n'}\) is
    % \begin{itemize}
    %     \item For all \(i \neq w\), \(e_{ij} \mapsto e_{ij}'\).
    %     \item For \(j = 1, ..., \lambda_w - r\), \(e_{wj} \mapsto e_{wj}'\).
    %     \item For \(j = 1, ..., r\), \(e_{wj} \mapsto f'_{n + j}\).
    %     \item For \(j = 1, ..., n\), \(f_j \mapsto f'_j - e_{w,\lambda_w - n + r}\)
    % \end{itemize}
    Change of basis by \(Q_{wr}\) maps \(\mathcal{Q}'\) to \(\mathcal{Q}_{wr}\), and change of basis by \(Q_{wr}^{-1}\) maps \(\mathcal{Q}_{wr}\) to \(\mathcal{Q}'\).
\end{proof}
The next section finds the irreducible components of \(\mathcal{Q}'\).
To make the notation nicer, we will refer to it as \(\mathcal{Q}\), and refer to \(m',n'\) as \(m,n\) (and so on), throughout the next section.

\section{The components of \texorpdfstring{\(\mathcal{Q}\)}{Q}}\label{q_comp}
\subsection{Setup}
Let \(X = J(\lambda) \in \gl_m\).
Let \((e_{ij})_{1 \leq i \leq k, 1 \leq j \leq \lambda_i}\) be the standard (Jordan) basis for \(\mathbb{C}^m\).
Let
\[\mathcal{Q} = \{(A_{0,b}, V_\bullet) : \forall i. \; A_{0,b} V_{i + 1} \subseteq V_i\}.\]
In this section we find the irreducible components of \(\mathcal{Q}\).
\par We write \(b_{ij}\) to denote the projection of \(b \in \mathbb{C}^m\) onto \(e_{ij}\).
For each row \(i\), let \(p_i(b) = \max\{j : b_{ij} \neq 0\}\) (the maximum of the empty set is zero).
Then set \(q_i(b) = \lambda_i - p_i(b)\).
When it is clear enough from context where the \(b\) is coming from, we will just write \(p_i\) and \(q_i\) instead of \(p_i(b)\) and \(q_i(b)\).
\par Let \(I = \{i_1 < \cdots < i_r\} \subseteq \{1, ..., k\}\), and let \((\rho_i)_{i \in I}\) be any map \(I \to \mathbb{N}_{> 0}\) such that (1) \(\rho_i \leq \lambda_i\), (2) \(\rho_i\) is decreasing with \(i\), (3) \(\lambda_i - \rho_i\) is decreasing with \(i\), and (4) \(\rho_i < n\).
For notational convenience (although we assign meaning to neither \(i_0\) nor \(i_{r + 1}\)), we define \(q_{i_0} := n\), and \(p_{i_{r + 1}} := 0\).
Then, we define \(B_{I, (\rho_i)}\) as the set of \(b \in \mathbb{C}^m\) satisfying the following conditions.
\begin{itemize}
    \item For all \(k \in \{1, ..., r\}\), \(p_i = \rho_i\).
    \item For all \(k \in \{0, ..., r\}\), \(p_{i_{k + 1}} = \max_{i : q_i < q_{i_k}} p_i\).
\end{itemize}
Note that for any \(b \in B_{I, (\rho_i)}\) we have \(p_{i_1} > \cdots > p_{i_r} > p_{i_{r + 1}}\), and also \(q_{i_0} > q_{i_1} > \cdots > q_{i_r}\).

\begin{lemma}\label{bs_union}
    \(\mathbb{C}^m = \bigcup_{I, (\rho_i)} B_{I, (\rho_i)}\), where \(I\) ranges over all subsets of \(\{1, ..., k\}\), and \((\rho_i)\) ranges over all maps \(I \to \mathbb{N}_{>0}\) satisfying the conditions (1),(2),(3),(4).
    Further, none of the \(B_{I, (\rho_i)}\) is contained in the union of the others.
\end{lemma}
\begin{proof}
    Let \(b \in \mathbb{C}^m\).
    If \(\{i : q_{i_0} > q_i\}\) is the empty set, then stop.
    Otherwise, take any \(i_1 \in \argmax_{i : q_{i_0} > q_i} p_i\), and set \(\rho_{i_1} := p_{i_1}\).
    If \(\{i : q_{i_1} > q_i\} = \emptyset\), then stop.
    Otherwise, take any \(i_2 \in \argmax_{i : q_{i_1} > q_i} p_i\), and set \(\rho_{i_2} := p_{i_2}\).
    Continuing on in this way, eventually we reach a \(k\) where \(\{i : q_{i_k} > q_i\} = \emptyset\).
    Then we set \(I = \{i_1, ..., i_k\}\).
    Note that \(I, (\rho_i)\) satisfy conditions (1)--(4), and furthermore \(b \in B_{I, (\rho_i)}\).
    \par Now, we check that no \(B_{I, (\rho_i)}\) is contained in the union of the others.
    Fix \(I\) and \((\rho_i)\).
    Take any \(b \in B_{I, (\rho_i)}\) with \(p_i = \rho_i\) for \(i \in I\) and \(p_i = 0\) for \(i \notin I\).
    It is clear that \(b \notin B_{I', (\rho_i')}\) whenever \(I' \neq I\) or \((\rho_i') \neq (\rho_i)\).
\end{proof}
\par Let \(\mathcal{Q}_{I, (\rho_i)} := \{(A_{0, b}, V_\bullet) \in \mathcal{Q} : b \in B_{I, (\rho_i)}\}\).
We will show that \(\mathcal{Q}_{I, (\rho_i)} \cong B_{I, (\rho_i)} \times (\textrm{some springer fiber})\).
Then we will use the result about the irreducible components of a Springer fiber to find the irreducible components of \(\mathcal{Q}_{I, (\rho_i)}\), and the closures of these will be the irreducible components of \(\mathcal{Q}\).

\subsection{Study of \texorpdfstring{\(B_{I, (\rho_i)}\)}{B\_\{I, (p\_i)\}}}
Fix any \(I\) and \((\rho_i)\) satisfying the conditions (1)--(4) of \cref{bs_union}.
As before, we write \(\{i_1 < \cdots < i_r\} := I\), and \(q_{i_0} := n\), and \(p_{i_{r + 1}} := 0\).

\par First, we provide an alternative characterization of \(B_{I, (\rho_i)}\).
\begin{lemma}\label{alternative_bs_one}
    \(B_{I, (\rho_i)}\) is the set of \(b \in \mathbb{C}^m\) satisfying the following conditions.
    \begin{itemize}
        \item For all \(k \in \{1, ..., r\}\), \(p_{i_k} = \rho_{i_k}\).
        \item For all \(i \notin I\),
        \begin{itemize}
            \item For all \(k \in \{0, ..., r\}\) such that \(\lambda_i > q_{i_k} + p_{i_{k + 1}}\), we have \(q_{i_k} \leq q_i\).
            \item For all \(k \in \{1, ..., r + 1\}\) such that \(\lambda_i \leq q_{i_{k - 1}} + p_{i_k}\), we have \(p_i \leq p_{i_k}\).
        \end{itemize}
    \end{itemize}
\end{lemma}
\begin{proof}
    First we show that every element of \(B_{I, (\rho_i)}\) satisfies those conditions.
    Let \(b \in B_{I, (\rho_i)}\).
    It is clear that \(\forall k. \; p_{i_k} = \rho_{i_k}\).
    \par Take \(i \notin I\) and \(k \in \{0, ..., r\}\) such that \(\lambda_i > q_{i_k} + p_{i_{k + 1}}\).
    Suppose for contradiction that \(q_i < q_{i_k}\).
    Then \(p_i \leq \max_{j : q_j < q_{i_k}} p_j = p_{i_{k + 1}}\).
    Then \(\lambda_i = p_i + q_i < q_{i_k} + p_{i_{k + 1}}\), a contradiction.
    So we must have \(q_{i_k} \leq q_i\), as desired.
    \par Now take \(i \notin I\) and \(k \in \{1, ..., r + 1\}\) such that \(\lambda_i \leq q_{i_{k - 1}} + p_{i_k}\).
    Suppose for contradiction that \(p_i > p_{i_k}\).
    Then, putting this together with the first inequality, \(\lambda_i - p_i < q_{i_{k - 1}} + p_{i_k} - p_{i_k}\); that is, \(q_i < q_{i_{k - 1}}\).
    Consequently, \(p_i \leq \max_{j : q_j < q_{i_{k - 1}}} p_j = p_{i_k}\), as desired.
    \par Now we have shown that every element of \(B_{I, (\rho_i)}\) satisfies the conditions of the lemma, and we proceed to the converse.
    Let \(b \in \mathbb{C}^m\) satisfy the conditions.
    Let \(k \in \{0, ..., r\}\).
    We need to show that \(\max_{j : q_j < q_{i_k}} p_j = p_{i_{k + 1}}\).
    Given the conditions (1)--(4) on \(\rho_i\), it suffices to show that for each \(i \notin I\) with \(q_i < q_{i_k}\), we have \(p_i \leq p_{i_{k + 1}}\).
    Indeed, given \(i \notin I\) with \(q_i < q_{i_k}\), we cannot have \(\lambda_i > q_{i_k} + p_{i_{k + 1}}\), as that would imply (by hypothesis) that \(q_{i_k} \leq q_i\).
    Hence \(\lambda_i \leq q_{i_{k - 1}} + p_{i_k}\), and consequently (by hypothesis) \(p_i \leq p_{i_k}\).
\end{proof}

\begin{corollary}\label{alternative_bs_two}
    \(B_{I, (\rho_i)}\) is the set of \(b \in \mathbb{C}^m\) satisfying the following conditions.
    \begin{itemize}
        \item For all \(k \in \{1, ..., r\}\), \(p_{i_k} = \rho_{i_k}\).
        \item For \(i \notin I\),
        \begin{itemize}
            \item If \(\lambda_i \geq q_{i_0} + p_{i_1}\), then \(p_i \leq \lambda_i - q_{i_0}\).
            \item If there is \(k \in \{1, ..., r\}\) with \(q_{i_{k - 1}} + p_{i_k} > \lambda_i \geq q_{i_k} + p_{i_{k + 1}}\), then \(p_i \leq \min(p_{i_k}, \lambda_i - q_{i_k})\).
            \item If \(q_{i_r} + p_{i_{r + 1}} > \lambda_i\), then \(p_i \leq p_{i_{r + 1}}\).
        \end{itemize}
    \end{itemize}
\end{corollary}
\begin{proof}
    Both \(q_{i_k}\) and \(p_{i_k}\) are decreasing as \(k\) increases, so this follows directly from \cref{alternative_bs_one}.
    (Note that \(p_i \leq \lambda_i - q_{i_k}\) iff \(q_{i_k} \leq q_i\).)
\end{proof}

\begin{corollary}\label{bs_iso}
    \begin{align*}
        B_{I, (\rho_i)} \cong & \prod_{k = 1}^r (\mathbb{C}^{\rho_{i_k} - 1} \times (\mathbb{C} \setminus \{0\})) \times \prod_{i : \lambda_i \geq q_{i_0} + p_{i_1}} \mathbb{C}^{\lambda_i - q_{i_0}} \times \\
        & \prod_{k = 1}^r \prod_{\{i \notin I : q_{i_{k - 1}} + p_{i_k} > \lambda_i \geq q_{i_k} + p_{i_{k + 1}}\}}\mathbb{C}^{\min(p_{i_k}, \lambda_i - q_{i_k})}.
    \end{align*}
\end{corollary}
\begin{proof}
    Here we use the notation \(x \times y = (x, y)\), and so on.
    The isomorphism sends \(b \in B_{I, (\rho_i)}\) to 
    \[\prod_{k = 1}^{r}(b_{i_k1},..., b_{i_k\rho_{i_k}}) \times \prod_{i : \lambda_i \geq q_{i_0} + p_{i_1}} (b_{i1}, ..., b_{i, \lambda_i - q_{i_0}}) \times\]
    \[\prod_{k = 1}^r \prod_{\{i \notin I : q_{i_{k - 1}} + p_{i_k} > \lambda_i \geq q_{i_k} + p_{i_{k + 1}}\}} (b_{i1}, ..., b_{i,\min(p_{i_k}, \lambda_i - q_{i_k})}).\]
    \cref{alternative_bs_two} says that this is an isomorphism.
\end{proof}

\begin{corollary}\label{bs_dim}
    \[\dim B_{I, (\rho_i)} = \sum_{i : \lambda_i \geq q_{i_0} + \rho_{i_1}} (\lambda_i - q_{i_0}) + \sum_{k = 1}^r \sum_{\{i : q_{i_{k - 1}} + p_{i_k} > \lambda_i \geq q_{i_k} + p_{i_{k + 1}}\}} \min(p_{i_k}, \lambda_i - q_{i_k}).\]
\end{corollary}
\begin{proof}
    Immediate from \cref{bs_iso}.
\end{proof}

\subsection{Study of \texorpdfstring{\(\mathcal{Q}_{I, (\rho_i)}\)}{Q\_\{I, (p\_i)\}}}
Fix any \(I\) and \((\rho_i)\) satisfying the conditions (1)--(4) of \cref{bs_union}.
In this subsection we find the irreducible components of \(\mathcal{Q}_{I, (\rho_i)}\).
\par We claim that there is some \(\mu(I, (\rho_i))\) such that every \((A_{0, b}, U) \in \mathcal{Q}_{I, (\rho_i)}\) is similar to \(J(\mu)\).
By finding a algebraic map taking \(b \in B_{I, (\rho_i)}\) to a Jordan basis for \(A_{0,b}\), we will put \(\mathcal{Q}_{I, (\rho_i)}\) in isomorphism with the product \(B_{I, (\rho_i)} \times (Springer fiber at J(\mu))\).
Then we will use our result about the usual Springer fiber at \(J(\mu)\) to find the irreducible components of \(\mathcal{Q}_{I, (\rho_i)}\).
\par So, now we find a Jordan basis for \(A_{0, b}\).
As before, we write \(\{i_1, ..., i_r\} := I\), and \(q_{i_0} := n\), and \(p_{i_{r + 1}} := 0\).
We also write, somewhat abusively, \(b \in \mathbb{C}^{m + n}\) to refer to the vector \((b, 0) \in \mathbb{C}^{m + n}\).
\begin{lemma}
    The following vectors give a Jordan basis for \(A_{0,b}\).
    (For convenience, we write \(A := A_{0,b}\) in this lemma and proof.)
    \begin{itemize}
        \item For \(i \notin I\), the chain of length \(p_i + q_i\) beginning with \(e_{i, p_i + q_i}\)
        \item The chain of length \(n + p_{i_1}\) beginning with \(f_n - (A^{n + p_{i_1}}f_n \rightshift n + p_{i_1})\)
        \item For \(k \in \{1, ..., r\}\), the chain of length \(q_{i_k} + p_{i_{k + 1}}\) beginning with \(v_{i_k} - (A^{q_{i_k} + p_{i_{k + 1}}} v_{i_k} \rightshift q_{i_k} + p_{i_{k + 1}})\), where \(v_{i_k} := A^{n - q_{i_k}} f_n - \sum_{l = 1}^{p_{i_k}} b_{i_kl} e_{i,l + q_{i_k}}\)
    \end{itemize}
\end{lemma}
\begin{proof}
    There are three things to check: (1) the chains are no longer than their claimed lengths, (2) the claimed lengths sum to \(m + n\), and (3) the span of the chains is \(\mathbb{C}^{m + n}\).
    \begin{proof}[Proof of (1)]
        It is obvious that a chain beginning with \(e_{i, p_i + q_i}\) has length \(p_i + q_i\).
        \par Now consider the chain beginning with \(f_n - (A^{n + p_{i_1}} f_n \rightshift n + p_{i_1})\).
        Note that \(A^n f_n = b\), so \(A^{n + p_{i_1}} f_n = A^{p_{i_1}} b = b \leftshift p_{i_1}\).
        By shifting \(b\) left \(p_{i_1}\) times, we zero out all the rows \(i\) where \(q_i < n\).
        This ensures that the operation of shifting \(b \leftshift p_{i_1}\) right \(n + p_{i_1}\) times is invertible by shifting left \(n + p_{i_1}\) times.
        That is,
        \begin{align*}
            A^{n + p_{i_1}} f_n & = \\
            b \leftshift p_{i_1} & = \\
            ((b \leftshift p_{i_1}) \rightshift n + p_{i_1}) \leftshift n + p_{i_1} & = \\
            A^{n + p_{i_1}}(A^{n + p_{i_1}}f_n \rightshift n + p_{i_1}) & .
        \end{align*}
        This shows that the chain has length at most \(n + p_{i_1}\), as desired.
        \par Now, let \(k \in \{1, ..., r\}\).
        We have \(v_{i_k} = A^{n - q_{i_k}} f_n - \sum_{l = 1}^{p_{i_k}} b_{i_kl} e_{i_k,l + q_{i_k}}\).
        First we note that \(q_{i_k} < n\) by the definition of \(\mathcal{Q}_{I, (\rho_i)}\), so the definition of \(v_{i_k}\) makes sense.
        We consider the chain beginning with \(v_{i_k} - (A^{q_{i_k} + p_{i_{k + 1}}} v_{i_k} \rightshift q_{i_k} + p_{i_{k + 1}})\).
        Note that \(A^{q_{i_k}} v_{i_k}\) is just \(b\) with row \(i_k\) zeroed out.
        For brevity, we write \(b_{i_k} := A^{q_{i_k}} v_{i_k}\).
        Note that \(b_{i_k} \leftshift p_{i_{k + 1}}\) has rows \(l\) zeroed out, for all \(l\) with \(q_l < q_{i_k}\).
        This ensures that shifting \(b_{i_k} \leftshift p_{i_{k + 1}}\) right \(q_{i_k} + p_{i_{k + 1}}\) times can be inverted by shifting left \(q_{i_k} + p_{i_{k + 1}}\) times.
        That is,
        \begin{align*}
            A^{q_{i_k} + p_{i_{k + 1}}} v_{i_k} & = \\
            b_{i_k} \leftshift p_{i_{k + 1}} & = \\
            ((b_{i_k} \leftshift p_{i_{k + 1}}) \rightshift q_{i_k} + p_{i_{k + 1}}) \leftshift q_{i_k} + p_{i_{k + 1}} & = \\
            A^{q_{i_k} + p_{i_{k + 1}}}(A^{q_{i_k} + p_{i_{k + 1}}} v_{i_k} \rightshift q_{i_k} + p_{i_{k + 1}}) & .
        \end{align*}
        This shows that the chain has length at most \(q_{i_k} + p_{i_{k + 1}}\), as desired.
    \end{proof}
    \begin{proof}[Proof of (2)]
        The sum of the lengths is
        \[\sum_{i \notin I} (p_i + q_i) + \sum_{k = 0}^r (q_{i_k} + p_{i_{k + 1}}) = \sum_i (q_i + p_i) = m + n.\]
    \end{proof}
    \begin{proof}[Proof of (3)]
        Let \(W\) be the span of the chains listed.
        We need to show that \(W = \mathbb{C}^{m + n}\).
        Because every \(i \in I\) satisfies \(q_i < n\), clearly \(\langle e_{ij}\rangle_{i,j : q_i \geq n} \subseteq W\).
        \par We claim that \(f_n \in W\) as well.
        To see this, consider the chain beginning with \(f_n - (A^{n + p_{i_1}} f_n \rightshift n + p_{i_1})\).
        As explained in the proof of (1), we have \(A^{n + p_{i_1}} f_n \in \langle e_{ij}\rangle_{i,j : q_i \geq n}\).
        Consequently, \((A^{n + p_{i_1}} f_n \rightshift n + p_{i_1}) \in \langle e_{ij} \rangle_{i,j : q_i \geq n} \subseteq W\).
        Because \(f_n - (A^{n + p_{i_1}} f_n \rightshift n + p_{i_1}) \in W\), this implies that \(f_n \in W\).
        \par Because \(AW \subseteq W\) (obvious, since \(W\) is a span of chains), the fact that \(f_n \in W\) implies that \(f_i \in W\) for each \(i\), and also \(b \leftshift l \in W\) for each \(l \geq 0\).
        \par Now we are left with showing that \(\langle e_{ij} \rangle_{i,j : q_i < n} \subseteq W\).
        It suffices to show that \(e_{i, \lambda_i} \in W\) for each \(i\) with \(q_i < n\).
        This is obvious for \(i \notin I\).
        So, we just need to show that \(e_{i_k\lambda_{i_k}} \in W\) for each \(k \in \{1, ..., r\}\).
        We do this inductively; fix \(k\), and suppose we have already shown that \(e_{i_{k'}, \lambda_{i_{k'}}} \in W\) for \(k' < k\).
        We will show that \(e_{i_k, \lambda_{i_k}} \in W\).
        \par To see this, consider the chain beginning with \(v_{i_k} - (A^{q_{i_k} + p_{i_{k + 1}}} v_{i_k} \rightshift q_{i_k} + p_{i_{k + 1}})\).
        (Recall \(v_{i_k} = A^{n - q_{i_k}} f_n - \sum_{l = 1}^{p_{i_k}} b_{i_kl} e_{i_k,l + q_{i_k}}\).)
        Since \(b_{i_k,p_{i_k}} \neq 0\) (by definition of \(p_{i_k}\)), it suffices to show that \(\sum_{l = 1}^{p_{i_k}} b_{i_kl} e_{i_k,l + q_{i_k}} \in W\).
        As explained in the proof of (1), we have \(A^{q_{i_k} + p_{i_{k + 1}}} v_{i_k} \in \langle e_{lj} \rangle_{q_l \geq q_{i_k}}\).
        And since \(A^{q_{i_k} + p_{i_{k + 1}}} v_{i_k}\) has row \(i_k\) zeroed out, in fact \(A^{q_{i_k} + p_{i_{k + 1}}} v_{i_k} \in \langle e_{lj} \rangle_{l : l \neq i_k \land q_l \geq q_{i_k}}\).
        Hence, \(A^{q_i + P_i} v_i \rightshift q_i + P_i \in \langle e_{lj} \rangle_{l : l \neq i \land q_l \geq q_i}\).
        By our inductive hypothesis, \(\langle e_{lj} \rangle_{l : l \neq i_k \land q_l \geq q_{i_k}} \subseteq W\), and consequently \(A^{q_{i_k} + p_{i_{k + 1}}} v_{i_k} \rightshift q_{i_k} + p_{i_{k + 1}} \in W\).
        Since we know \(v_{i_k} - (A^{q_{i_k} + p_{i_{k + 1}}} v_{i_k} \rightshift q_{i_k} + p_{i_{k + 1}}) \in W\), this implies that \(v_{i_k} \in W\).
        Because \(A^{n - q_{i_k}} f_n \in W\), this then implies that \(\sum_{l = 1}^{p_{i_k}} b_{i_kl} e_{i_k,l + q_{i_k}} \in W\), as desired.
    \end{proof}
    We proved (1), (2), (3), so we are done.
\end{proof}

Let \(\mu(I, (\rho_i))\) be the shape of the Jordan basis given in the lemma.
Let \(X_\mu\) be the Springer fiber at \(J(\mu)\), and let \((X_{\mu, \alpha})_{\alpha \in \mathrm{SYT}(\mu)}\) be the irreducible components.
\par Given a zero-indexed list \(L = [L_0, ..., L_{l - 1}]\), we define \(\gamma(L) = \sum_i i L_i\).
We are interested in this thing because for any \(\alpha\), the dimension of \(X_{\mu, \alpha}\) is \(\gamma([\mu_1, ..., \mu_{k + 1}])\).

\begin{lemma}\label{springer_fiber_dim}
    \[\gamma([\mu_1, ..., \mu_{k + 1}]) = \gamma([0, \lambda_1, ..., \lambda_k]) - \] 
    \[\left[\sum_{i : \lambda_i \geq q_{i_0} + \rho_{i_1}} (\lambda_i - q_{i_0}) + \sum_{k = 1}^r \sum_{\{i : q_{i_{k - 1}} + p_{i_k} > \lambda_i \geq q_{i_k} + p_{i_{k + 1}}\}} \min(p_{i_k}, \lambda_i - q_{i_k})\right].\]
\end{lemma}
\begin{proof}
    Let \(L = [q_{i_0}, \lambda_1, ..., \lambda_k]\).
    Note that
    \[\mu = [..., q_{i_0} + p_{i_1}, ..., q_{i_1} + p_{i_2}, ..., ..., q_{i_r} + p_{i_{r + 1}}, ...].\]
    Let \(L'\) be the result of taking \(\mu\) and, for each \(x\), replacing one occurence of \(q_{i_x} + p_{i_{x + 1}}\) by \(q_{i_x} + p_{i_x}\); that is,
    \[L' = [..., q_{i_0}, ..., q_{i_1} + p_{i_1}, ..., q_{i_2} + p_{i_2}, ..., q_{i_r} + p_{i_r}].\]
    We can transform \(\mu\) into \(L'\) by just `moving' each \(p_{i_k}\) to the right by \(1 + \#\{i \notin I : q_{i_{k - 1}} + p_{i_k} > \lambda_i \geq q_{i_k} + p_{i_{k + 1}}\}\) slots.
    So,
    \[\gamma(L') - \gamma(\mu) = \sum_{k = 1}^r p_{i_k} (1 + \#\{i \notin I : q_{i_{k - 1}} + p_{i_k} > \lambda_i \geq q_{i_k} + p_{i_{k + 1}}\}) = \]
    \[\sum_{k = 1}^r p_{i_k} \cdot \#\{i : q_{i_{k - 1}} + p_{i_k} > \lambda_i \geq q_{i_k} + p_{i_{k + 1}}\}.\]
    \par Now we consider how to transform \(L'\) into \(L\).
    First we shift \(q_{i_0}\) to the left by \(\#\{i \notin I : \lambda_i \geq q_{i_0} + p_{i_1}\}\) slots.
    Then we leave \(q_{i_0}\) in place and sort the rest of the list.
    This entails shifting each \(q_{i_x} + p_{i_x}\) to the left by \(\#\{i \notin I : q_{i_x} + p_{i_x} > \lambda_i \geq q_{i_x} + p_{i_{x + 1}}\}\) slots.
    Shifting \(q_{i_x} + p_{i_x}\) to the left one slot, by swapping it with \(\lambda_i\), changes the value of \(\gamma\) by \(\lambda_i - (q_{i_x} + p_{i_x})\).
    To go from \(L'\) to \(L\), we can just make these swaps repeatedly.  
    So,
    \[\gamma(L) - \gamma(L') = \]
    \[\sum_{i \notin I : \lambda_i \geq q_{i_0} + p_{i_1}} (\lambda_i - q_{i_0}) + \sum_{k = 1}^s \sum_{\{i \notin I : q_{i_k} + p_{i_k} > \lambda_i \geq q_{i_k} + p_{i_{k + 1}}\}} (\lambda_i - (q_{i_k} + p_{i_k})).\]
    \par Now, we put these two results together to get \(\gamma(L) - \gamma(\mu)\).
    \begin{align*}
        \gamma(L) - \gamma(\mu) & = \\
        [\gamma(L) - \gamma(L')] + [\gamma(L') - \gamma(\mu)] & = \\
        \left[\sum_{i : \lambda_i \geq q_{i_0} + p_{i_1}} (\lambda_i - q_{i_0}) + \sum_{k = 1}^s \sum_{\{i : q_{i_k} + p_{i_k} > \lambda_i \geq q_{i_k} + p_{i_{k + 1}}\}} (\lambda_i - (q_{i_k} + p_{i_k}))\right] & + \\
        \left[\sum_{k = 1}^s \sum_{\{i : q_{i_{k - 1}} + p_{i_k} > \lambda_i \geq q_{i_k} + p_{i_{k + 1}}\}} p_{i_k}\right] & = \\
        \sum_{i : \lambda_i \geq q_{i_0} + p_{i_1}} (\lambda_i - q_{i_0}) + \sum_{k = 1}^s \sum_{\{i : q_{i_{k - 1}} + p_{i_k} > \lambda_i \geq q_{i_k} + p_{i_k}\}} p_{i_k} & + \\
        \sum_{k = 1}^s \sum_{i : q_{i_k} + p_{i_k} > \lambda_i \geq q_{i_k} + p_{i_{k + 1}}} (\lambda_i - q_{i_k}) & = \\
        \sum_{i : \lambda_i \geq q_{i_0} + p_{i_1}} (\lambda_i - q_{i_0}) + \sum_{k = 1}^s \sum_{\{i : q_{i_{k - 1}} + p_{i_k} > \lambda_i \geq q_{i_k} + p_{i_{k + 1}}\}} \min(p_{i_k}, \lambda_i - q_{i_k}) & .
    \end{align*}
    Since \(\gamma(L) = \gamma([0, \lambda_1, ..., \lambda_k])\), the equation above implies the desired result.
\end{proof}

\begin{lemma}\label{u_i_rho_iso}
    \(\mathcal{Q}_{I, (\rho_i)} \cong B_{I, (\rho_i)} \times X_\mu\).
\end{lemma}
\begin{proof}
    For \(b \in B_{I, (\rho_i)}\), let \(P_b\) be the change-of-basis matrix, with columns given by the Jordan basis of the previous lemma, so that \(J(\mu) = P_b^{-1} A_{0,b} P_b\).
    From looking at the Jordan basis, it is clear that the map \(P : B_{I, (\rho_i)} \to \GL_{m + n}\), given by \(b \mapsto P_b\), is algebraic.
    \par Now, we remark that the Springer fiber at \(A_{0,b}\) is simply \(\{P_b V_\bullet : V_\bullet \in X_\mu\}\).
    This gives the isomorphism \(B_{I, (\rho_i)} \times X_\mu \to \mathcal{Q}_{I, (\rho_i)}\) defined by
    \[(b, V_\bullet) \mapsto (P_b J(\mu) P_b^{-1}, P_b V_\bullet),\]
    with inverse
    \[(A_{0,b}, V_\bullet) \mapsto (b, P_b^{-1} V_\bullet).\]
\end{proof}

\begin{corollary}\label{u_i_rho_irred_and_dim}
    For \(\alpha \in \mathrm{SYT}(\mu)\), let \(\mathcal{Q}_{I, (\rho_i), \alpha}\) be the subvariety of \(\mathcal{Q}_{I, (\rho_i)}\) which corresponds to \(B_{I, (\rho_i)} \times X_{\mu, \alpha}\) via the isomorphism of \cref{u_i_rho_iso}.
    The subvarieties \((\mathcal{Q}_{I, (\rho_i), \alpha})_{\alpha \in \mathrm{SYT}(\mu)}\) are the irreducible components of \(\mathcal{Q}_{I, (\rho_i)}\).
    Each has dimension \(\gamma([0, \lambda_1, ..., \lambda_k])\).
\end{corollary}
\begin{proof}
    We know that \(B_{I, (\rho_i)}\) is irreducible by \cref{bs_iso}.
    Then, the fact that the \(B_{I, (\rho_i)} \times X_{\mu, \alpha}\) are the irreducible components of \(B_{I, (\rho_i)} \times X_\mu\) just follows from the fact that the \(X_{\mu, \alpha}\) are the irreducible components of \(X_\mu\).
    \par To get the dimension, we add the dimension of \(B_{I, (\rho_i)}\) to the dimension of \(X_{\mu, \alpha}\).
    We get the dimension of \(B_{I, (\rho_i)}\) from \cref{bs_dim}, and we get the dimension of \(X_{\mu, \alpha}\) from \cref{springer_fiber_dim}.
    Adding them together, things cancel out and we get \(\gamma([0, \lambda_1, ..., \lambda_k])\).
\end{proof}

\subsection{Conclusion}
\begin{theorem}\label{Q_comps}
    The irreducible components of \(\mathcal{Q}\) are the closures of the subvarieties \(\mathcal{Q}_{I, (\rho_i), \alpha}\), as we let \(I, (\rho_i)\) range over all possibilities satisfying the conditions (1)--(4) of \cref{bs_union}, and we let \(\alpha \in \mathrm{SYT}(\mu(I, (\rho_i)))\).
\end{theorem}
\begin{proof}
    By \cref{u_i_rho_irred_and_dim}, the \(\mathcal{Q}_{I, (\rho_i), \alpha}\) are irreducible and equidimensional.
    Then \cref{bs_union} says that their union is \(\mathcal{Q}\), and in addition that none is contained in the union of the others.
\end{proof}

\section{The components of \texorpdfstring{\(\mathcal{P}_{J(\lambda)}\)}{P\_\{J(\textbackslash lambda)\}}}\label{p_comp}
As in section 5, we write \(X = J(\lambda) \in \gl_m\), where \(\lambda = (\lambda_1, ..., \lambda_k)\).
We write \(\mathcal{P} := \mathcal{P}_{J(\lambda)}\), and \(A_{a,b} := A_{J(\lambda), a, b}\).
And as before, \((e_{ij})_{ij}\) and \((f_j)_j\) form the standard basis for \(\mathbb{C}^{m + n}\).
\subsection{The components of \texorpdfstring{\(\mathcal{Q}_{wr}\)}{Q\_\{wr\}}}
Recall from section 5 the varieties
\[\mathcal{Q}_{wr} = \{(A_{e_{wr}, b}, V_\bullet) \in \mathcal{P}\}.\]
Fix any \(w,r\).
\cref{a_zero} tells us that 
\[\mathcal{Q}_{wr} \cong \mathcal{Q}' := \{(A'_{X', 0, b}, V_\bullet) : \forall i. \; A'_{X', 0, b} V_{i + 1} \subseteq V_i\},\]
where \(X' = J(\lambda')\), and \(\lambda' = (\lambda_1, ..., \lambda_w - r, ..., \lambda_k)\), and \(m' = m - r\), and \(n' = n + r\).
\par Let \((\mathcal{Q}'_{I, (\rho_i), \alpha})_{I, (\rho_i), \alpha}\) be the irreducible components of \(\mathcal{Q}'\) give by \cref{Q_comps}.
Write \(\mathcal{Q}_{w,r,I,(\rho_i),\alpha}\) to denote the irreducible component of \(\mathcal{Q}_{wr}\) corresponding to \(\mathcal{Q}'_{I, (\rho_i), \alpha}\) via the isomorphism \(\mathcal{Q}_{wr} \cong \mathcal{Q}'\) of \cref{a_zero}.

\begin{theorem}\label{qwr_comps}
    The irreducible components of \(\mathcal{Q}_{wr}\) are the subvarieties \(\mathcal{Q}_{w,r,I,(\rho_i),\alpha}\).
    Each has dimension \(\sum_{i \leq j} \min(\lambda_i', \lambda_j')\), where \(\lambda' = (\lambda_1, ..., \lambda_{w - 1}, \lambda_w - r, \lambda_{w + 1}, ..., \lambda_k)\).
\end{theorem}
\begin{proof}
    From the foregoing discussion, it is clear that they are indeed the irreducible components.
    To calculate the dimension, we refer to \cref{u_i_rho_irred_and_dim}, which says the dimension is \(\gamma([0, \lambda_1', ..., \lambda_k'])\).
\end{proof}

\subsection{The varieties \texorpdfstring{\(G_{wr}\)}{G\_\{wr\}} and \texorpdfstring{\(G\)}{G}}
Fix \(w,r\).
Recall from section 5 the groups \(G = \{P \in \GL_m : P^{-1}XP = X\}\) and \(G_{wr} = \{A \in G : e_{wr} A = e_{wr}\}\).
\begin{lemma}\label{g_irred_dim}
    \(G\) is irreducible and has dimension \(\sum_{ij} \min(\lambda_i, \lambda_j)\).
\end{lemma}
\begin{proof}
    The closure of \(G\) in \(\gl_m\) is \(\mathfrak{z}_{\gl_m}(X)\).
    \cref{nilpotent_centralizer} says that \(\mathfrak{z}_{\gl_m}(X)\) is isomorphic to \(\mathbb{C}^{\sum_{ij} \min(\lambda_i, \lambda_j)}\).
\end{proof}

\begin{lemma}\label{gwr_irred_dim}
    \(G_{wr}\) is irreducible and has dimension \(\sum_{ij} \min(\lambda_i, \lambda_j')\), where \(\lambda' = (\lambda_1, ..., \lambda_{w - 1}, \lambda_w - r, \lambda_{w + 1}, ..., \lambda_k)\).
\end{lemma}
\begin{proof}
    The closure of \(G_{wr}\) in \(\gl_m\) is \(V = \{Y \in \mathfrak{z}_{\gl_m}(X) : e_{wr} Y = e_{wr}\}\).
    In the case that \(r = 0\), the constraint that \(e_{wr}Y = e_{wr}\) is no constraint at all, so we have \(G_{wr} = G\), and the result follows from \cref{g_irred_dim}.
    \par In the case that \(r > 0\), we observe that \(V = \{Y + I : Y \in \mathfrak{z}_{\gl_m}(X), e_{wr}Y = 0\}\).
    The constraint \(e_{wr}Y = 0\) is just saying that a certain row of \(Y\) must be all zeroes.
    So, the set of \(Y\) such that \(Y + I \in V\) is the set described by \cref{nilpotent_centralizer_row_zero}.
    Hence \(V \cong \mathbb{C}^{\sum_{ij}\min(\lambda_i', \lambda_j')}\).
\end{proof}

\subsection{The components of \texorpdfstring{\(\mathcal{P}_{wr}\)}{P\_\{wr\}}}
Recall from section 5 the subvarieties \(\mathcal{P}_{wr} \subseteq \mathcal{P}\).
From \cref{principal_bundle} we have the principal \(G_{wr}\)-bundle \(\varphi_{wr} : \mathcal{Q}_{wr} \times G \to \mathcal{P}_{wr}\).
% \par Together, \cref{g_irred_dim} and \cref{gwr_irred_dim} tell us that the irreducible components of \(\mathcal{Q}_{wr} \times G\) are the subvarieties \(\mathcal{Q}_{w,r,I,(\rho_i),\alpha} \times G\).

\begin{theorem}\label{pwr_comps}
    Every irreducible component of \(\mathcal{P}_{wr}\) is the closure of some subvariety of the form \(\varphi_{wr}(\mathcal{Q}_{w,r,I,(\rho_i),\alpha} \times G)\), and the closure of each subvariety of this form is an irreducible component.
    Each has dimension \(\sum_{i \leq j} \min(\lambda_i, \lambda_j)\).
\end{theorem}
\begin{remark}
    \cref{pwr_comps} does not say ``the irreducible components of \(\mathcal{P}_{wr}\) are the closures of the subvarieties \(\varphi_{wr}(\mathcal{Q}_{w,r,I,(\rho_i),\alpha} \times G)\)'', as that would seem to suggest some claim about distinctness.
    Further analysis is required to determine which ones are distinct.
\end{remark}
\begin{proof}
    Together, \cref{Q_comps} and \cref{g_irred_dim} tell us that the \(\mathcal{Q}_{w,r,I,(\rho_i),\alpha}\) are irreducible, and their union is \(\mathcal{Q}_{wr}\).
    Hence their images are irreducible, and the surjectivity of \(\varphi_{wr}\) implies that their union of their images is \(\mathcal{P}_{wr}\).
    \par Now, to verify that each image is an irreducible component, we need only verify that they are equidimensional.
    We use the result of \cref{principal_bundle}, namely that \(\varphi_{wr}\) is a principal \(G_{wr}\)-bundle.
    \par Let \(V_{w,r,I,(\rho_i),\alpha}\) be the closure of \(\mathcal{Q}_{w,r,I,(\rho_i),\alpha} \times G\) in \(\mathcal{Q}_{wr} \times G\) under action by \(G_{wr}\).
    Clearly \(\varphi_{wr}(\mathcal{Q}_{w,r,I,(\rho_i),\alpha} \times G) = \varphi_{wr}(V_{w,r,I,(\rho_i),\alpha})\).
    So, we have only to compute the dimension of \(\varphi(V_{w,r,I,(\rho_i),\alpha})\).
    \par This is easy, because \(\varphi_{wr}\) (and, consequently, the restriction of \(\varphi_{wr}\) to \(V_{w,r,I,(\rho_i),\alpha}\)) is a principal \(G_{wr}\)-bundle, and principal bundles play nicely with dimensions.
    That is, we can conclude that 
    \[\dim \varphi_{wr}(V_{w,r,I,(\rho_i),\alpha}) = \dim V_{w,r,I,(\rho_i),\alpha} - \dim G_{wr}.\]
    Since \(\dim (\mathcal{Q}_{w,r,I,(\rho_i),\alpha} \times G) = \dim (\mathcal{Q}_{wr} \times G)\) by \cref{qwr_comps}, we know that \(\dim V_{w,r,I,(\rho_i),\alpha} = \dim (\mathcal{Q}_{w,r,I,(\rho_i),\alpha} \times G) = \dim \mathcal{Q}_{w,r,I,(\rho_i),\alpha} + \dim G\).
    By the equation above then, we get 
    \[\dim \varphi_{wr}(V_{w,r,I,(\rho_i),\alpha}) = \dim \mathcal{Q}_{w,r,I,(\rho_i),\alpha} + \dim G - \dim G_{wr}.\]
    We already calculated those dimensions in \cref{qwr_comps}, \cref{g_irred_dim}, and \cref{gwr_irred_dim} respectively.
    Referring to those, we get 
    \begin{equation}\label{the_equation}
        \dim \varphi_{wr}(V_{w,r,I,(\rho_i),\alpha}) = \sum_{i \leq j} \min(\lambda_i', \lambda_j') + \sum_{ij} \min(\lambda_i, \lambda_j) - \sum_{ij} \min(\lambda_i, \lambda_j').
    \end{equation}
    Now,
    \[\sum_{i \leq j} \min(\lambda_i', \lambda_j') = \sum_{i \leq j} \min(\lambda_i, \lambda_j) - \sum_j \min(\lambda_w, \lambda_j) + \sum_j \min(\lambda_w - r, \lambda_j),\]
    and
    \[\sum_{ij} \min(\lambda_i, \lambda_j') = \sum_{ij} \min(\lambda_i, \lambda_j) - \sum_i \min(\lambda_i, \lambda_w) + \sum_i \min(\lambda_i, \lambda_w - r).\]
    Substituting these into the RHS of \cref{the_equation}, things cancel out, and we get
    \[\sum_{i \leq j} \min(\lambda_i, \lambda_j) + \sum_{ij} \min(\lambda_i, \lambda_j) - \sum_{ij} \min(\lambda_i, \lambda_j) = \sum_{i \leq j} \min(\lambda_i, \lambda_j).\]
\end{proof}

\subsection{The components of \texorpdfstring{\(\mathcal{P}\)}{P}}
\begin{corollary}\label{p_comps}
    Every irreducible component of \(\mathcal{P}\) is the closure of some subvariety of the form \(\varphi_{wr}(\mathcal{Q}_{w,r,I,(\rho_i),\alpha} \times G)\), and the closure of each subvariety of this form is an irreducible component.
    Each has dimension \(\sum_{i \leq j} \min(\lambda_i, \lambda_j)\).
\end{corollary}
\begin{proof}
    Follows directly from \cref{pwr_comps}.
\end{proof}

\section{The components of \texorpdfstring{\(S \times_\mathfrak{g} \widetilde{\mathcal{N}}\)}{S x\_g N}}\label{sxn_comps}
Recall from \cref{sn_iso} that 
\[S \times_\mathfrak{g} \widetilde{\mathcal{N}} \cong \{(A_{X,a,b}, U_\bullet, V_\bullet) : \forall i. \; XU_{i + 1} \subseteq U_i; \forall i. \; A_{X,a,b} V_{i + 1} \subseteq V_i\}.\]
We defined \(\pi : S \times_\mathfrak{g} \widetilde{\mathcal{N}} \to \mathcal{N}_m\) by \((A_{X,a,b}, U_\bullet, V_\bullet) \mapsto X\), and called \(\pi^{-1}(X)\) the \(n\)-Slodowy-slice Springer fiber at \(X\).
\begin{theorem}\label{springer_fiber_comps}
    \(\pi^{-1}(J(\lambda)) \cong \mathcal{P}_{J(\lambda)} \times X_\lambda\).
    Each irreducible component of \(\pi^{-1}(J(\lambda))\) has dimension \(\sum_{ij} \min(\lambda_i, \lambda_j)\).
\end{theorem}
\begin{proof}
    The isomorphism sends \((A_{X,a,b}, U_\bullet, V_\bullet)\) to \(((A_{X, a, b}, U_\bullet), V_\bullet)\).
    The irreducible components of \(\mathcal{P}_{J(\lambda)}\) are given by \cref{p_comps}.
    Those of \(X_\lambda\) are given by \cref{usual_springer_fiber}.
    Adding the dimensions gives 
    \[\sum_{i \leq j} \min(\lambda_i, \lambda_j) + \sum_{i < j} \min(\lambda_i, \lambda_j) = \sum_{ij} \min(\lambda_i, \lambda_j).\]
\end{proof}

Let \(\lambda\) be any partition of \(m\).
Let \(\GL_m \times \{I_n\}\) act on \(\pi^{-1}(J(\lambda))\) by conjugation; that is,
\[(P, I_n) \cdot (A_{X, a, b}, U_\bullet, V_\bullet) := \]
\[((P,I_n)A_{X,a,b}(P,I_n)^{-1},(P,I_n)U_\bullet, (P,I_n)V_\bullet).\]
Let 
\[K = \{(g, I_n) \in \GL_m \times \{I_n\} : gXg^{-1} = X\}.\]

Define \(\phi_\lambda : \pi^{-1}(J(\lambda)) \times (\GL_m \times \{I_n\}) \to S \times_\mathfrak{g} \widetilde{\mathcal{N}}\) by \(\phi_\lambda(x, g) = g \cdot x\).

\begin{lemma}\label{principal_bundle_two}
    For each partition \(\lambda\) of \(m\), the map \(\phi_\lambda\) is a principal \(K\)-bundle.
\end{lemma}
\begin{proof}
    Analogous to \cref{principal_bundle}.
\end{proof}

Let \((C_{\lambda, \beta})_\beta\) be the irreducible components of \(\pi^{-1}(J(\lambda))\).
These are described by \cref{springer_fiber_comps}.

\begin{theorem}
    Every irreducible component of \(S \times_\mathfrak{g} \widetilde{\mathcal{N}}\) is the closure of some subvariety of the form \(\phi_\lambda(C_{\lambda,\beta} \times (\GL_m \times \{I_n\}))\), and the closure of each subvariety of this form is an irreducible component.
    Each has dimension \(m^2\).
\end{theorem}
\begin{proof}
    This is analogous to \cref{pwr_comps}; the only difference is the dimension calculation.
    This time, we have 
    \begin{align*}
        \dim \phi_\lambda(C_{\lambda, \beta} \times (\GL_m \times \{I_n\})) & = \\
        \dim C_{\lambda,\beta} + \dim \GL_m - \dim K & = \\
        \sum_{ij} \min(\lambda_i, \lambda_j) + m^2 - \sum_{ij} \min(\lambda_i, \lambda_j) & = \\
        m^2 & .
    \end{align*}
    The dimension of \(C_{\lambda, \beta}\) comes from \cref{pwr_comps}; the dimension of \(\GL_m\) is obvious; and the dimension of \(K\) comes from \cref{nilpotent_centralizer}.
\end{proof}

\section{Linear Algebra Facts}\label{linalg}
In this section we prove linear algebra facts that were used earlier.
They are confined to this section to avoid interrupting the rest of the paper.

\subsection{The centralizer of a nilpotent matrix}
\begin{definition}
    A matrix \(Y\) is \emph{Toeplitz} if it is constant along bands parallel to the main diagonal.
    That is, \(\forall i,j,k. \; Y_{ij} = Y_{i + k, j + k}\).
\end{definition}
\begin{definition}
    An \(m \times n\) matrix \(Y\) is \emph{lower-left Toeplitz} if it is Toeplitz and, in addition, we have \(y_{n - i, j - 1} = 0\) whenever \(i + j \geq \min(m,n)\).
\end{definition}
That is, \(Y\) is lower-left Toeplitz if it is Toeplitz, and the only nonzero entries are those with Manhattan distance less than \(\min(m,n)\) from the entry in the bottom-left corner.
In yet other words, all but the leftmost (equivalently, bottommost) \(\min(m,n)\) diagonal bands are zero.

\begin{lemma}\label{nilpotent_centralizer}
    Let \(\lambda = (\lambda_1, ..., \lambda_k)\) be a partition of \(m\).
    The centralizer of \(J(\lambda)\) in \(\gl_m\) is the subalgebra consisting of matrices 
    \[M = \begin{pmatrix}
        M_{11} & \cdots & M_{1k}\\
        \vdots &        & \vdots\\
        M_{k1} & \cdots & M_{kk}
    \end{pmatrix},\]
    where each \(M_{ij}\) is a \(\lambda_i \times \lambda_j\) matrix, such that each \(M_{ij}\) is lower-left Toeplitz.
\end{lemma}
\begin{proof}
    Let 
    \[M = \begin{pmatrix}
        M_{11} & \cdots & M_{1k}\\
        \vdots &        & \vdots\\
        M_{k1} & \cdots & M_{kk}
    \end{pmatrix}.\]
    We need to show that \(J(\lambda)M = MJ(\lambda)\) if and only if each \(M_{ij}\) is lower-left Toeplitz.
    \par We have 
    \[J(\lambda)M = \begin{pmatrix}
        J_{\lambda_1}M_{11} & \cdots & J_{\lambda_1}M_{1k}\\
        \vdots & & \vdots\\
        J_{\lambda_k}M_{k1} & \cdots & J_{\lambda_k}M_{kk}
    \end{pmatrix}, \textrm{ and } MJ(\lambda) = \begin{pmatrix}
        M_{11} J_{\lambda_1} & \cdots & M_{1k} J_{\lambda_k}\\
        \vdots & & \vdots\\
        M_{k1} J_{\lambda_1} & \cdots & M_{kk} J_{\lambda_k}
    \end{pmatrix}.\]
    So, we have \(J(\lambda)M = MJ(\lambda)\) if and only if \(\forall i,j. \; J_{\lambda_i} M_{ij} = M_{ij} J_{\lambda_j}\).
    Multiplying on the left by \(J_{\lambda_i}\) just shifts each row down by one, and multiplying on the right by \(J_{\lambda_j}\) shifts each column left by one.
    The matrices for which left-shifting gives the same result as down-shifting are exactly the lower-left Toeplitz matrices.
\end{proof}

\begin{corollary}\label{nilpotent_centralizer_row_zero}
    Let \(\lambda = (\lambda_1, ..., \lambda_k)\) be a partition of \(m\).
    Let \(w \in \{1, ..., k\}\) and \(r \in \{1, ..., \lambda_w\}\).
    Let \(i = [\sum_{j < w} \lambda_j] + r\).
    The set of \(M \in \mathfrak{z}_{\gl_m}(J(\lambda))\) such that the \(i\)th row of \(M\) is equal to zero is the set of matrices
    \[M = \begin{pmatrix}
        M_{11} & \cdots & M_{1k} \\
        \vdots & & \vdots\\
        M_{k1} & \cdots & M_{kk}
    \end{pmatrix},\]
    where each \(M_{ij}\) is a \(\lambda_i \times \lambda_j\) matrix, such that:
    \begin{itemize}
        \item For \(i \neq w\), \(M_{ij}\) is lower-left Toeplitz
        \item Each \(M_{wj}\) is of the form
        \[M_{wj} = \begin{pmatrix}
            0\\
            M_{wj}'
        \end{pmatrix},\]
        where \(M_{wj}'\) is a \((\lambda_w - r) \times \lambda_j\) lower-left Toeplitz matrix.
    \end{itemize}
\end{corollary}

\subsection{A `normalization' fact about Jordan bases}
\begin{lemma}\label{normalization_helper_one}
    For any finite-dimensional \(V\), nilpotent \(A : V \to V\), and linear \(f : V \to \mathbb{C}\), there is a Jordan basis \(e_{ij}\) for \(A\) such that there is at most one \(i\) such that there exists \(j\) such that \(f(e_{ij}) \neq 0\).
\end{lemma}
\begin{proof}
    For any Jordan basis \((e_{ij})_{ij}\) of \(A\), define
    \[S((e_{ij})_{1 \leq i \leq k, 1 \leq j \leq \lambda_i}) := \sum_i \begin{cases}-1, & \forall j. \; f (e_{ij}) = 0 \\ \lambda_i - \min\{j : f (e_{ij}) \neq 0\}, & \textrm{otherwise} \end{cases}.\]
    We proceed by induction on the measure \(S\).
    That is, let \((e_{ij})_{ij}\) be a Jordan basis for \(A\).
    Our inductive hypothesis is that if there exists a Jordan basis \((e_{ij}')_{ij}\) with \(S((e_{ij}')_{ij}) < S((e_{ij})_{ij})\), then we get the desired conclusion.
    \par Now, we have two cases.
    In the first case, \((e_{ij})_{ij}\) already satisfies the desired property.
    In this case we are done.
    In the other case, there exist \(i_1, j_1, i_2, j_2\) with \(i_1 \neq i_2\), and \(f(e_{i_1j_1}) \neq 0\), and \(f(e_{i_2j_2}) \neq 0\).
    We let \(j_1, j_2\) be minimal with this property, so that \(\forall j < j_1. \; f(e_{i_1j}) = 0\), and \(\forall j < j_2. \; f(e_{i_2j}) = 0\).
    Wlog, we assume that \(\lambda_{i_1} - j_1 \leq \lambda_{i_2} - j_2\).
    \par By our inductive hypothesis, all we need to do is find a Jordan basis \((e_{ij}')_{ij}\) with \(S((e_{ij}')_{ij}) < S((e_{ij})_{ij})\).
    This is what we do.
    Define \(e'_{ij}\) as follows.
    \begin{itemize}
        \item \(e_{i_1, \lambda_1}' := e_{i_1, \lambda_1} - \frac{f(e_{i_1, j_1})}{f(e_{i_2, j_2})}e_{i_2, j_2 + (\lambda_{i_1} - j_1)}\)
        \item For \(j < \lambda_1\), \(e_{i_1, j}' := A^{\lambda_{i_1} - j} e_{i_1, \lambda_{i_1}}'\)
        \item For \(i \neq i_1\), \(e_{ij}' := e_{ij}\).
    \end{itemize}
    Clearly this is a Jordan basis for \(A\).
    Further, we claim that \(S((e'_{ij})_{ij})  < S((e_{ij})_{ij})\).
    It suffices to show that \(\forall j \leq j_1. \; f (e_{i_1,j}) = 0\).
    We have 
    \[f (e_{i_1,j}') = f\left(A^{\lambda_{i_1} - j}\left(e_{i_1, \lambda_1} - \frac{f(e_{i_1, j_1})}{f(e_{i_2, j_2})}e_{i_2, j_2 + (\lambda_{i_1} - j_1)}\right)\right) =\]
    \[f\left(e_{i_1, j} - \frac{f(e_{i_1, j_1})}{f(e_{i_2, j_2})} e_{i_2, j_2 + (j - j_1)}\right) = f(e_{i_1, j}) - \frac{f(e_{i_1, j_1})}{f(e_{i_2, j_2})} f(e_{i_2, j_2 + (j - j_1)}).\]
    Clearly (by design), this expression is zero when \(j = j_1\).
    And for \(j < j_1\), we have \(f(e_{i_1, j}) = f(e_{i_2, j_2 + (j - j_1)}) = 0\), so it is zero then as well.
    Hence the measure \(S\) of this new basis is smaller, as desired.
\end{proof}

\begin{lemma}\label{normalization_helper_two}
    For any \(n\) and linear \(f : \mathbb{C}^n \to \mathbb{C}\), there is a Jordan basis \(e_j\) for \(J_n\) such that there is at most one \(j\) with \(f(e_j) \neq 0\).
\end{lemma}
\begin{proof}
    Let \(e_j\) be a Jordan basis for \(J_n\).
    If \(\{j : f(e_j) \neq 0\}\) is the empty set, we are done.
    Otherwise, let \(j_0 = \min\{j : f(e_j) \neq 0\}\).
    For any Jordan basis \(f_j\) with \(j_0 = \min\{j : f(e_j) \neq 0\}\), define 
    \[S((f_j)_j) := \begin{cases}
        -1, & \{j > j_0 : f(e_j) \neq 0\} = \emptyset \\
        n - \min\{j > j_0 : f(e_j) \neq 0\}, & \textrm{otherwise}
    \end{cases}.\]
    We proceed by induction on \(S\).
    That is, let \((e_j)_j\) be a Jordan basis for \(J_n\) with \(j_0 = \min\{j : f(e_j) \neq 0\}\).
    Our inductive hypothesis is that if there exists a Jordan basis \((e_j')_j\) with \(j_0 = \min\{j : f(e_j') \neq 0\}\) and \(S((e_j')_j) < S((e_j)_j)\), then the conclusion holds.
    \par We have two cases: either \((e_j)_j\) satisfies the desired property, or not.
    If not, then let \(j_1 = \min\{j > j_0 : f(e_j) \neq 0\}\), and define a new Jordan basis \(e_j'\) as follows.
    \begin{itemize}
        \item \(e_n' := e_n - \frac{f(e_{j_1})}{f(e_{j_0})} e_{n - (j_1 - j_0)}\)
        \item For \(j < n\), \(e_j' := J_n^{n - j} e_n'\)
    \end{itemize}
    It is straightforward to check that \(j_0 = \min\{j : f(e_j') \neq 0\}\), and that \(S((e_j')_j) \leq S((e_j)_j) - 1\).
    By our inductive hypothesis, we are done.
\end{proof}

\begin{theorem}\label{normalization}
    For any finite-dimensional \(V\), nilpotent \(A : V \to V\), and linear \(f : V \to \mathbb{C}\), there is a Jordan basis \(e_{ij}\) for \(A\) such that there is at most one pair \((i, j)\) with \(f(e_{ij}) \neq 0\).
\end{theorem}
\begin{proof}
    \cref{normalization_helper_one} provides a Jordan basis \(e_{ij}\) such that for all \(i \neq i_0\) and all \(j\), we have \(f(e_{ij}) = 0\).
    Restricting \(A\) to \(\langle e_{i_0j}\rangle_{j \textrm{ arbitrary}}\) gives a Jordan block, and then applying \cref{normalization_helper_two} gives the desired result.
\end{proof}

\subsection{Nilpotency Lemmas}
\begin{lemma}\label{upper_triangle_zero}
    Let \(X \in \gl_n\) be upper triangular.
    Then \(J_n + X\) is nilpotent if and only if \(X = 0\).
\end{lemma}
\begin{proof}
    Clearly if \(X = 0\), then \(J_n + X\) is nilpotent.
    Inversely, suppose \(X \neq 0\).
    Let \(e_1, ..., e_n\) be the standard basis, with \(J_n e_i = e_{i + 1}\).
    Let \(i_1 = \max\{i : Xe_i \neq 0\}\).
    As \(X\) is upper triangular, we have \(X e_{i_1} = v + ae_{i_2}\), with \(a \in \mathbb{C}\setminus\{0\}\), \(i_2 \leq i_1\), and \(v \in \langle e_1, ..., e_{i_1 - 1}\rangle\).
    \par Now, \((J_n + X)^{i_1} e_1 = e_{i_1 + 1} + v + ae_{i_2}\).
    Then, \((J_n + X)^{i_1 + (n - i_1)} e_1 = 0 + (J_n + X)^{n - i_1} (v + ae_{i_2})\).
    Clearly \((J_n + X)^{n - i_1} (v + ae_{i_2}) = v' + ae_{i_2 + n - i_1}\), with \(v' \in \langle e_1, ..., e_{i_2 + n - i_1 - 1} \rangle\).
    Now, since \(i_2 \leq i_1\), we have \(i_2 + n - i_1 \leq n\), and therefore \((J_n + X)^n e_1 \neq 0\).
    It follows that \(J_n + X\) is not nilpotent.
\end{proof}

\begin{lemma}\label{block_determinant}
    Let \(X \in \gl_m\), and let 
    \[Y = \begin{pmatrix}
        y_{11} & y_{12} & y_{13} & \cdots & y_{1,n-1} & y_{1n} \\
        d_1 & y_{22} & y_{23} & \cdots & y_{2,n-1} & y_{2n} \\
            & d_2   & y_{33} & \cdots & y_{3,n-1} & y_{3n} \\
            & & \ddots & \cdots & \vdots  & \vdots \\
            & & & d_{n - 2} & y_{n-1,n-1} & y_{n - 1,n}\\
            & & & & d_{n - 1} & y_{nn}
    \end{pmatrix} \in \gl_n.\]
    For any \(a, b \in \mathbb{C}^m\),
    \[\det 
    \begin{pNiceArray}{ccc|ccc}
        & & & & & \vert \\
        & X & & & & b    \\
        & & & & & \vert \\
       \hline
       \text{---} & a & \text{---} &  \\
       & & & & Y & \\
       & & &  \\
       \end{pNiceArray} = \]
       \[\det X \det Y + \left(\prod_i d_i\right) \det \begin{pNiceArray}{ccc|c}
            & & & \vert \\
            & X & & b     \\
            & & & \vert \\
            \hline 
        \text{---} & a & \text{---} & 0
       \end{pNiceArray}\]
\end{lemma}
\begin{proof}
    By induction on \(n\).
    In the case \(n = 1\), expanding along the last row (taking the usual interpretation of the empty product) gives the desired result.
    \par Now suppose \(n > 1\).
    Expanding along the last row, we get 
    \[d_{n - 1} \det \begin{pNiceArray}{ccc|ccc}
        & & & & & \vert \\
        & X & & & & b    \\
        & & & & & \vert \\
       \hline
       \text{---} & a & \text{---} &  \\
       & & & & Y_{n,n-1} & \\
       & & &  \\
       \end{pNiceArray} -
    y_{nn} \det \begin{pNiceArray}{ccc|ccc}
        & & & & & \\
        & X & & & &   \\
        & & & & & \\
       \hline
       \text{---} & a & \text{---} &  \\
       & & & & Y_{n,n} & \\
       & & &  \\
       \end{pNiceArray}.\]
    Using our inductive hypothesis for the first determiniant, and using that \(\det \begin{pNiceArray}{c|c}
        A_{11} & 0 \\
        A_{21} & A_{22}
    \end{pNiceArray} = \det A_{11} \det A_{22}\) for the second, the expression becomes
    \[d_{n - 1} \left(\det X \det Y_{n,n-1} + \left(\prod_{i \leq n - 2} d_i\right) \det \begin{pNiceArray}{ccc|c}
        & & & \vert \\
        & X & & b     \\
        & & & \vert \\
        \hline 
    \text{---} & a & \text{---} & 0
   \end{pNiceArray}\right)
   - y_{nn} \det X \det Y_{nn} = \]
   \[(d_{n - 1} Y_{n,n-1} - y_{nn} \det Y_{nn})\det X + \left(\prod_i d_i\right) \det \begin{pNiceArray}{ccc|c}
        & & & \vert \\
        & X & & b     \\
        & & & \vert \\
        \hline 
    \text{---} & a & \text{---} & 0
   \end{pNiceArray} = \]
   \[\det Y \det X + \left(\prod_i d_i\right) \det \begin{pNiceArray}{ccc|c}
        & & & \vert \\
        & X & & b     \\
        & & & \vert \\
        \hline 
    \text{---} & a & \text{---} & 0
   \end{pNiceArray}.\]
\end{proof}

\begin{corollary}\label{bottom_right_nilp}
    If \(X\) is nilpotent, and
    \[\begin{pNiceArray}{ccc|ccc}
        & & & & & \vert \\
        & X & & & & b    \\
        & & & & & \vert \\
       \hline
       \text{---} & a & \text{---} &  \\
       & & & & Y & \\
       & & &  \\
       \end{pNiceArray}\]
    is nilpotent  as well, then \(Y\) is nilpotent.
%     If all the \(d_i\) are nonzero, then
%     \[\begin{pNiceArray}{ccc|c}
%         & & & \vert \\
%         & X & & b     \\
%         & & & \vert \\
%         \hline 
%     \text{---} & a & \text{---} & 0
%    \end{pNiceArray}\]
%    is nilpotent as well.
% ^That is false.  I want to say something about its characteristic polynomial.  not sure that I care enough to take the space though.
    \begin{proof}
        By the previous lemma, the characteristic polynomial of the big matrix is
        \[g_X(\lambda) g_Y(\lambda) + f(\lambda),\]
        where \(g_X(\lambda) = \lambda^m\) is the characteristic polynomial of \(X\), and \(f(\lambda)\) is some polynomial of degree at most \(m - 1\).
    \end{proof}
\end{corollary}

\bibliography{paper}
\end{document}
